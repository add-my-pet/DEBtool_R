\nonstopmode{}
\documentclass[a4paper]{book}
\usepackage[times,inconsolata,hyper]{Rd}
\usepackage{makeidx}
\usepackage[utf8,latin1]{inputenc}
% \usepackage{graphicx} % @USE GRAPHICX@
\makeindex{}
\begin{document}
\chapter*{}
\begin{center}
{\textbf{\huge Package `DEBtoolAnimal'}}
\par\bigskip{\large \today}
\end{center}
\begin{description}
\raggedright{}
\item[Type]\AsIs{Package}
\item[Title]\AsIs{DEB functions fot an animal}
\item[Version]\AsIs{0.1}
\item[Date]\AsIs{2015-09-30}
\item[Author]\AsIs{Goncalo M. Marques }\email{goncalo.marques@tecnico.ulisboa.pt}\AsIs{}
\item[Maintainer]\AsIs{Goncalo M. Marques }\email{goncalo.marques@tecnico.ulisboa.pt}\AsIs{}
\item[Description]\AsIs{DEB based functions for the std and abj models for animals.}
\item[License]\AsIs{GPL}
\item[LazyData]\AsIs{TRUE}
\item[NeedsCompilation]\AsIs{no}
\end{description}
\Rdcontents{\R{} topics documented:}
\inputencoding{utf8}
\HeaderA{beta0}{Particular incomplete beta function}{beta0}
%
\begin{Description}\relax
particular incomplete beta function:
\end{Description}
%
\begin{Usage}
\begin{verbatim}
beta0(x0, x1)
\end{verbatim}
\end{Usage}
%
\begin{Arguments}
\begin{ldescription}
\item[\code{x0}] scalar with lower boundary for integration

\item[\code{x1}] scalar with upper boundary for integration
\end{ldescription}
\end{Arguments}
%
\begin{Value}
scalar with particular incomple beta function
\end{Value}
%
\begin{SeeAlso}\relax
Other miscellaneous functions: \code{\LinkA{C2K}{C2K}};
\code{\LinkA{K2C}{K2C}}
\end{SeeAlso}
%
\begin{Examples}
\begin{ExampleCode}
beta0(0.1, 0.2)
\end{ExampleCode}
\end{Examples}
\inputencoding{utf8}
\HeaderA{C2K}{Conversion of Celsius to Kelvin}{C2K}
%
\begin{Description}\relax
Computes Kelvin from temperatures defined in Celsius
\end{Description}
%
\begin{Usage}
\begin{verbatim}
C2K(C)
\end{verbatim}
\end{Usage}
%
\begin{Arguments}
\begin{ldescription}
\item[\code{C}] numeric temperature in degrees Celsius
\end{ldescription}
\end{Arguments}
%
\begin{Value}
temperature in Kelvin
\end{Value}
%
\begin{SeeAlso}\relax
Other miscellaneous functions: \code{\LinkA{K2C}{K2C}};
\code{\LinkA{beta0}{beta0}}
\end{SeeAlso}
%
\begin{Examples}
\begin{ExampleCode}
C2K(20)
\end{ExampleCode}
\end{Examples}
\inputencoding{utf8}
\HeaderA{dget\_lbarb2}{Computes derivative d delta/dx}{dget.Rul.lbarb2}
%
\begin{Description}\relax
Obtains the derivative d delta/dx from lbarb, xb and k.
\end{Description}
%
\begin{Usage}
\begin{verbatim}
dget_lbarb2(x, delta, pars)
\end{verbatim}
\end{Usage}
%
\begin{Arguments}
\begin{ldescription}
\item[\code{x}] scalar x = g/(g + e)

\item[\code{delta}] scalar delta = x e\_H/ (1 - kap)g

\item[\code{pars}] data.frame with lbarb, xb, xb3 (xb\textasciicircum{}1/3), k
\end{ldescription}
\end{Arguments}
%
\begin{Value}
scalar with derivative value d delta/ dx
\end{Value}
%
\begin{SeeAlso}\relax
Other scaled get functions: \code{\LinkA{fnget\_lbarb2}{fnget.Rul.lbarb2}};
\code{\LinkA{get\_lbarb2}{get.Rul.lbarb2}}; \code{\LinkA{get\_lbarb}{get.Rul.lbarb}};
\code{\LinkA{get\_lb}{get.Rul.lb}};
\code{\LinkA{initial\_scaled\_reserve}{initial.Rul.scaled.Rul.reserve}}
\end{SeeAlso}
%
\begin{Examples}
\begin{ExampleCode}
dget_lbarb2(10^(-6), 0, c(lbarb = 0.003, xb = 10/11, xb3 = (10/11)^(1/3), k = 1))
\end{ExampleCode}
\end{Examples}
\inputencoding{utf8}
\HeaderA{fnget\_lbarb2}{Computes f using the ode solver for delta(x), for finding lbarb}{fnget.Rul.lbarb2}
%
\begin{Description}\relax
Computes f using the ode solver for delta(x), for finding lbarb.
\end{Description}
%
\begin{Usage}
\begin{verbatim}
fnget_lbarb2(lbarb, pars)
\end{verbatim}
\end{Usage}
%
\begin{Arguments}
\begin{ldescription}
\item[\code{lbarb}] scalar with scaled length at birth (lbarb = lb/ g)

\item[\code{pars}] data.frame with lbarb, xb, xb3 (xb\textasciicircum{}1/3), k
\end{ldescription}
\end{Arguments}
%
\begin{Value}
scalar with function f which when zero indicates lbarb
\end{Value}
%
\begin{SeeAlso}\relax
Other scaled get functions: \code{\LinkA{dget\_lbarb2}{dget.Rul.lbarb2}};
\code{\LinkA{get\_lbarb2}{get.Rul.lbarb2}}; \code{\LinkA{get\_lbarb}{get.Rul.lbarb}};
\code{\LinkA{get\_lb}{get.Rul.lb}};
\code{\LinkA{initial\_scaled\_reserve}{initial.Rul.scaled.Rul.reserve}}
\end{SeeAlso}
%
\begin{Examples}
\begin{ExampleCode}
fnget_lbarb2(0.03, c(xb = 10/11, xb3 = (10/11)^(1/3), vbarHb = 0.001, k = 1))
\end{ExampleCode}
\end{Examples}
\inputencoding{utf8}
\HeaderA{get\_lb}{Computes scaled length at birth}{get.Rul.lb}
%
\begin{Description}\relax
Obtains scaled length at birth, given the scaled reserve density at birth.
\end{Description}
%
\begin{Usage}
\begin{verbatim}
get_lb(pars, eb = 1, lb0 = as.numeric(pars[3]^(1/3)))
\end{verbatim}
\end{Usage}
%
\begin{Arguments}
\begin{ldescription}
\item[\code{pars}] 3-vector with parameters: g, k, v\_H\textasciicircum{}b

\item[\code{eb}] optional scalar with scaled reserve density at birth (default eb = 1)

\item[\code{lbarb0}] optional scalar with initial estimate for scaled length at birth (default lb0: lb for k = 1)
\end{ldescription}
\end{Arguments}
%
\begin{Value}
scalar with scaled length at birth (lb) and indicator equals 1 if successful, 0 otherwise (info)
\end{Value}
%
\begin{SeeAlso}\relax
Other scaled get functions: \code{\LinkA{dget\_lbarb2}{dget.Rul.lbarb2}};
\code{\LinkA{fnget\_lbarb2}{fnget.Rul.lbarb2}}; \code{\LinkA{get\_lbarb2}{get.Rul.lbarb2}};
\code{\LinkA{get\_lbarb}{get.Rul.lbarb}};
\code{\LinkA{initial\_scaled\_reserve}{initial.Rul.scaled.Rul.reserve}}
\end{SeeAlso}
%
\begin{Examples}
\begin{ExampleCode}
get_lb(c(g = 10, k = 1, vHb = 0.5), 1)
\end{ExampleCode}
\end{Examples}
\inputencoding{utf8}
\HeaderA{get\_lbarb}{Computes scaled length at birth lbarb}{get.Rul.lbarb}
%
\begin{Description}\relax
Obtains scaled length at birth, given the scaled reserve density at birth.
\end{Description}
%
\begin{Usage}
\begin{verbatim}
get_lbarb(pars, eb = 1, lbarb0 = NA)
\end{verbatim}
\end{Usage}
%
\begin{Arguments}
\begin{ldescription}
\item[\code{pars}] 3-vector with parameters: g, k, vbar\_H\textasciicircum{}b

\item[\code{eb}] optional scalar with scaled reserve density at birth (default eb = 1)

\item[\code{lbarb0}] optional scalar with initial estimate for scaled length at birth (default lbarb0: lbarb for k = 1)
\end{ldescription}
\end{Arguments}
%
\begin{Value}
scalar with scaled length at birth (lbarb) and indicator equals 1 if successful, 0 otherwise (info)
\end{Value}
%
\begin{SeeAlso}\relax
Other scaled get functions: \code{\LinkA{dget\_lbarb2}{dget.Rul.lbarb2}};
\code{\LinkA{fnget\_lbarb2}{fnget.Rul.lbarb2}}; \code{\LinkA{get\_lbarb2}{get.Rul.lbarb2}};
\code{\LinkA{get\_lb}{get.Rul.lb}};
\code{\LinkA{initial\_scaled\_reserve}{initial.Rul.scaled.Rul.reserve}}
\end{SeeAlso}
%
\begin{Examples}
\begin{ExampleCode}
get_lbarb(c(g = 10, k = 1, vbarHb = 0.0005), 1)
\end{ExampleCode}
\end{Examples}
\inputencoding{utf8}
\HeaderA{get\_lbarb2}{Computes initial scaled reserve}{get.Rul.lbarb2}
%
\begin{Description}\relax
Obtains scaled length at birth, given the scaled reserve density at birth. Like get\_lbarb, but uses a shooting method in 1 variable.
\end{Description}
%
\begin{Usage}
\begin{verbatim}
get_lbarb2(pars, eb = NA)
\end{verbatim}
\end{Usage}
%
\begin{Arguments}
\begin{ldescription}
\item[\code{pars}] 3-vector with parameters: g, k, vbar\_H\textasciicircum{}b

\item[\code{eb}] optional scalar with scaled reserve density at birth (default eb = 1)
\end{ldescription}
\end{Arguments}
%
\begin{Value}
scalar with scaled length at birth (lbarb) and indicator equals 1 if successful, 0 otherwise (info)
\end{Value}
%
\begin{SeeAlso}\relax
Other scaled get functions: \code{\LinkA{dget\_lbarb2}{dget.Rul.lbarb2}};
\code{\LinkA{fnget\_lbarb2}{fnget.Rul.lbarb2}}; \code{\LinkA{get\_lbarb}{get.Rul.lbarb}};
\code{\LinkA{get\_lb}{get.Rul.lb}};
\code{\LinkA{initial\_scaled\_reserve}{initial.Rul.scaled.Rul.reserve}}
\end{SeeAlso}
%
\begin{Examples}
\begin{ExampleCode}
get_lbarb2(c(g = 10, k = 1, vbarHb = 0.01), 1)
\end{ExampleCode}
\end{Examples}
\inputencoding{utf8}
\HeaderA{get\_ubarE0}{Computes initial scaled reserve density at birth}{get.Rul.ubarE0}
%
\begin{Description}\relax
Obtains the initial scaled reserve given the scaled reserve density at birth.
Function get\_ue0 does so for eggs, get\_ue0\_foetus for foetuses.
Specification of length at birth as third input by-passes its computation,
so if you want to specify an initial value for this quantity, you should use get\_lb directly.
\end{Description}
%
\begin{Usage}
\begin{verbatim}
get_ubarE0(g = NA, k = NA, vbarHb = NA, eb = 1, lbarb = NA)
\end{verbatim}
\end{Usage}
%
\begin{Arguments}
\begin{ldescription}
\item[\code{eb:}] optional scalar with scaled reserbe density at birth

\item[\code{x1}] scalar with upper boundary for integration
\end{ldescription}
\end{Arguments}
%
\begin{Value}
scalar with particular incomple beta function
\end{Value}
%
\begin{SeeAlso}\relax
Other get functions: \code{\LinkA{get\_ue0}{get.Rul.ue0}}
\end{SeeAlso}
%
\begin{Examples}
\begin{ExampleCode}
get_ubarE0(g = 10, lbarb = 0.01)
get_ubarE0(g = 10, k = 0.7, vbarHb = 5e-4)
\end{ExampleCode}
\end{Examples}
\inputencoding{utf8}
\HeaderA{get\_ue0}{Computes initial scaled reserve}{get.Rul.ue0}
%
\begin{Description}\relax
Obtains the initial scaled reserve given the scaled reserve density at birth.
Function get\_ue0 does so for eggs, get\_ue0\_foetus for foetuses.
Specification of length at birth as third input by-passes its computation,
so if you want to specify an initial value for this quantity, you should use get\_lb directly.
\end{Description}
%
\begin{Usage}
\begin{verbatim}
get_ue0(pars, eb = 1, lb0 = NA)
\end{verbatim}
\end{Usage}
%
\begin{Arguments}
\begin{ldescription}
\item[\code{pars}] 1 or 3 -vector with parameters g, k\_J/ k\_M, v\_H\textasciicircum{}b, see get\_lb

\item[\code{eb}] optional scalar with scaled reserbe density at birth (default: eb = 1)

\item[\code{lb0}] optional scalar with scaled length at birth (default: lb is optained from get\_lb)
\end{ldescription}
\end{Arguments}
%
\begin{Value}
uE0 scalar with scaled reserve at t=0: \$U\_E\textasciicircum{}0 g\textasciicircum{}2 k\_M\textasciicircum{}3/ v\textasciicircum{}2\$ with \$U\_E\textasciicircum{}0 = M\_E\textasciicircum{}0/ \{J\_EAm\}\$, lb scalar with scaled length at birth and info indicator equals 1 if successful, 0 otherwise
\end{Value}
%
\begin{SeeAlso}\relax
Other get functions: \code{\LinkA{get\_ubarE0}{get.Rul.ubarE0}}
\end{SeeAlso}
%
\begin{Examples}
\begin{ExampleCode}
get_ue0(pars = c(0.42, 1, 0.066), eb = 1, lb0 = 0.4042)
\end{ExampleCode}
\end{Examples}
\inputencoding{utf8}
\HeaderA{initial\_scaled\_reserve}{Gets initial scaled reserve}{initial.Rul.scaled.Rul.reserve}
%
\begin{Description}\relax
Gets initial scaled reserve.
\end{Description}
%
\begin{Usage}
\begin{verbatim}
initial_scaled_reserve(f, pars, Lb0 = NA)
\end{verbatim}
\end{Usage}
%
\begin{Arguments}
\begin{ldescription}
\item[\code{f}] n-vector with scaled functional responses

\item[\code{pars}] 5-vector with parameters: VHb, g, kJ, kM, v

\item[\code{Lb0}] optional n-vector with lengths at birth
\end{ldescription}
\end{Arguments}
%
\begin{Value}
n-vector with initial scaled reserve: M\_E\textasciicircum{}0/ J\_EAm (U0), n-vector with length at birth (Lb) and n-vector with 1's if successful, 0's otherwise (info)
\end{Value}
%
\begin{SeeAlso}\relax
Other scaled get functions: \code{\LinkA{dget\_lbarb2}{dget.Rul.lbarb2}};
\code{\LinkA{fnget\_lbarb2}{fnget.Rul.lbarb2}}; \code{\LinkA{get\_lbarb2}{get.Rul.lbarb2}};
\code{\LinkA{get\_lbarb}{get.Rul.lbarb}}; \code{\LinkA{get\_lb}{get.Rul.lb}}
\end{SeeAlso}
%
\begin{Examples}
\begin{ExampleCode}
initial_scaled_reserve(f = c(1, 0.9), pars = c(VHb = .8, g = .42, kJ = 1.7, kM = 1.7, v = 3.24))
\end{ExampleCode}
\end{Examples}
\inputencoding{utf8}
\HeaderA{K2C}{Conversion of Kelvin to Celsius}{K2C}
%
\begin{Description}\relax
Computes Celsius from temperatures given in Kelvin
\end{Description}
%
\begin{Usage}
\begin{verbatim}
K2C(K)
\end{verbatim}
\end{Usage}
%
\begin{Arguments}
\begin{ldescription}
\item[\code{K}] numeric temperature in degrees Kelvin
\end{ldescription}
\end{Arguments}
%
\begin{Value}
temperature in Kelvin
\end{Value}
%
\begin{SeeAlso}\relax
Other miscellaneous functions: \code{\LinkA{C2K}{C2K}};
\code{\LinkA{beta0}{beta0}}
\end{SeeAlso}
%
\begin{Examples}
\begin{ExampleCode}
K2C(293.15)
\end{ExampleCode}
\end{Examples}
\inputencoding{utf8}
\HeaderA{tempcorr}{Conversion of Kelvin to Celsius}{tempcorr}
%
\begin{Description}\relax
Calculates the factor with which physiological rates should be multiplied to go from a reference temperature to a given temperature
\end{Description}
%
\begin{Usage}
\begin{verbatim}
tempcorr(Temp, T_1, Tpars)
\end{verbatim}
\end{Usage}
%
\begin{Arguments}
\begin{ldescription}
\item[\code{Temp}] vector with new temperatures

\item[\code{T\_1}] scalar with reference temperature

\item[\code{Tpars}] 1-, 3- or 5-vector with temperature parameters
\end{ldescription}
\end{Arguments}
%
\begin{Details}\relax
This is a test \eqn{\theta}{}
\deqn{\dot{\theta}(T) = \dot{\theta}(T_1) \exp\left(\frac{T_A}{T_1} - \frac{T_A}{T}\right)}{}
\end{Details}
%
\begin{Value}
vector with temperature correction factors that affect all rates
\end{Value}
%
\begin{Examples}
\begin{ExampleCode}
tempcorr(c(330, 331, 332), 320, c(12000, 277, 318, 20000, 190000))
\end{ExampleCode}
\end{Examples}
\printindex{}
\end{document}
