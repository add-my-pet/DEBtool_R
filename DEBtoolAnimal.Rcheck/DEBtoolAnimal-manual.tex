\nonstopmode{}
\documentclass[a4paper]{book}
\usepackage[times,inconsolata,hyper]{Rd}
\usepackage{makeidx}
\usepackage[utf8,latin1]{inputenc}
% \usepackage{graphicx} % @USE GRAPHICX@
\makeindex{}
\begin{document}
\chapter*{}
\begin{center}
{\textbf{\huge Package `DEBtoolAnimal'}}
\par\bigskip{\large \today}
\end{center}
\begin{description}
\raggedright{}
\item[Type]\AsIs{Package}
\item[Title]\AsIs{DEB functions for an animal}
\item[Version]\AsIs{0.1}
\item[Date]\AsIs{2015-09-30}
\item[Author]\AsIs{Goncalo M. Marques }\email{goncalo.marques@tecnico.ulisboa.pt}\AsIs{}
\item[Maintainer]\AsIs{Goncalo M. Marques }\email{goncalo.marques@tecnico.ulisboa.pt}\AsIs{}
\item[Description]\AsIs{DEB based functions for the  std (standard) and abj (with acceleration) models for animals.}
\item[License]\AsIs{GPL}
\item[LazyData]\AsIs{TRUE}
\item[NeedsCompilation]\AsIs{no}
\end{description}
\Rdcontents{\R{} topics documented:}
\inputencoding{utf8}
\HeaderA{addchem}{Sets chemical parameters and text for units and labels}{addchem}
%
\begin{Description}\relax
Sets chemical parameters and text for units and labels
\end{Description}
%
\begin{Usage}
\begin{verbatim}
addchem(par, units, label, free, phylum, class)
\end{verbatim}
\end{Usage}
%
\begin{Arguments}
\begin{ldescription}
\item[\code{par}] data frame with parameter values

\item[\code{units}] data frame with parameter units

\item[\code{label}] data frame with parameter labels

\item[\code{free}] data frame with information on which parameter to free or fix

\item[\code{phylum}] string with species phylum

\item[\code{class}] string with species class
\end{ldescription}
\end{Arguments}
%
\begin{Details}\relax
Calls get\_d\_V to set specific density of structure. For a specific density of wet mass of 1 g/cm\textasciicircum{}3,
a specific density of d\_E = d\_V = 0.1 g/cm\textasciicircum{}3 means a dry-over-wet weight ratio of 0.1
\end{Details}
%
\begin{Value}
list with updated par, units, label and free
\end{Value}
%
\begin{SeeAlso}\relax
Other add-my-pet auxiliary functions: \code{\LinkA{addpseudodata}{addpseudodata}};
\code{\LinkA{fieldnm\_wtxt}{fieldnm.Rul.wtxt}}; \code{\LinkA{fieldnmnst}{fieldnmnst}};
\code{\LinkA{get\_d\_V}{get.Rul.d.Rul.V}}; \code{\LinkA{parscomp}{parscomp}};
\code{\LinkA{predict\_pseudodata}{predict.Rul.pseudodata}};
\code{\LinkA{print\_filterflag}{print.Rul.filterflag}};
\code{\LinkA{rmpseudodata}{rmpseudodata}}; \code{\LinkA{setweights}{setweights}}
\end{SeeAlso}
%
\begin{Examples}
\begin{ExampleCode}
pars_init_my_pet(metaData)
\end{ExampleCode}
\end{Examples}
\inputencoding{utf8}
\HeaderA{addpseudodata}{Adds pseudodata information into inputed data structures}{addpseudodata}
%
\begin{Description}\relax
Adds the pseudodata information and weights for purposes of the regression
\end{Description}
%
\begin{Usage}
\begin{verbatim}
addpseudodata(data = list(), units = list(), label = list(),
  weights = list())
\end{verbatim}
\end{Usage}
%
\begin{Arguments}
\begin{ldescription}
\item[\code{data}] structure with data

\item[\code{units}] structure with data units

\item[\code{label}] structure with data labels

\item[\code{weights}] structure with weights
\end{ldescription}
\end{Arguments}
%
\begin{Value}
structures with data, units, label and weights
\end{Value}
%
\begin{SeeAlso}\relax
Other add-my-pet auxiliary functions: \code{\LinkA{addchem}{addchem}};
\code{\LinkA{fieldnm\_wtxt}{fieldnm.Rul.wtxt}}; \code{\LinkA{fieldnmnst}{fieldnmnst}};
\code{\LinkA{get\_d\_V}{get.Rul.d.Rul.V}}; \code{\LinkA{parscomp}{parscomp}};
\code{\LinkA{predict\_pseudodata}{predict.Rul.pseudodata}};
\code{\LinkA{print\_filterflag}{print.Rul.filterflag}};
\code{\LinkA{rmpseudodata}{rmpseudodata}}; \code{\LinkA{setweights}{setweights}}
\end{SeeAlso}
%
\begin{Examples}
\begin{ExampleCode}
list[data, units, label, weight] <- addpseudodata();
\end{ExampleCode}
\end{Examples}
\inputencoding{utf8}
\HeaderA{beta0}{Particular incomplete beta function}{beta0}
%
\begin{Description}\relax
particular incomplete beta function:
\end{Description}
%
\begin{Usage}
\begin{verbatim}
beta0(x0, x1)
\end{verbatim}
\end{Usage}
%
\begin{Arguments}
\begin{ldescription}
\item[\code{x0}] scalar with lower boundary for integration

\item[\code{x1}] scalar with upper boundary for integration
\end{ldescription}
\end{Arguments}
%
\begin{Details}\relax
Computes
\deqn{B_{x_1}\left(\frac{4}{3},0\right) - B_{x_0}\left(\frac{4}{3},0\right) = \int_{x_0}^{x_1} t^{4/3-1} (1-t)^{-1} dt}{}
To be used in the computation of the age at birth (or related quantities) for an egg.
\end{Details}
%
\begin{Value}
scalar with particular incomplete beta function
\end{Value}
%
\begin{SeeAlso}\relax
Other miscellaneous functions: \code{\LinkA{C2K}{C2K}};
\code{\LinkA{K2C}{K2C}}
\end{SeeAlso}
%
\begin{Examples}
\begin{ExampleCode}
beta0(0.1, 0.2)
\end{ExampleCode}
\end{Examples}
\inputencoding{utf8}
\HeaderA{C2K}{Conversion of Celsius to Kelvin}{C2K}
%
\begin{Description}\relax
Converts temperature in degrees Celsius to Kelvin
\end{Description}
%
\begin{Usage}
\begin{verbatim}
C2K(C)
\end{verbatim}
\end{Usage}
%
\begin{Arguments}
\begin{ldescription}
\item[\code{C}] numeric temperature in degrees Celsius
\end{ldescription}
\end{Arguments}
%
\begin{Value}
temperature in Kelvin
\end{Value}
%
\begin{SeeAlso}\relax
Other miscellaneous functions: \code{\LinkA{K2C}{K2C}};
\code{\LinkA{beta0}{beta0}}
\end{SeeAlso}
%
\begin{Examples}
\begin{ExampleCode}
C2K(20)
\end{ExampleCode}
\end{Examples}
\inputencoding{utf8}
\HeaderA{dget\_lbarb2}{Computes derivative d delta/dx}{dget.Rul.lbarb2}
%
\begin{Description}\relax
Obtains the derivative d delta/dx from lbarb, xb and k.
\end{Description}
%
\begin{Usage}
\begin{verbatim}
dget_lbarb2(x, delta, pars)
\end{verbatim}
\end{Usage}
%
\begin{Arguments}
\begin{ldescription}
\item[\code{x}] scalar x = g/(g + e)

\item[\code{delta}] scalar delta = x e\_H/ (1 - kap)g

\item[\code{pars}] data.frame with lbarb, xb, xb3 (xb\textasciicircum{}1/3), k
\end{ldescription}
\end{Arguments}
%
\begin{Value}
scalar with derivative value d delta/ dx
\end{Value}
%
\begin{SeeAlso}\relax
Other scaled get functions: \code{\LinkA{fnget\_lbarb2}{fnget.Rul.lbarb2}};
\code{\LinkA{get\_lbarb2}{get.Rul.lbarb2}}; \code{\LinkA{get\_lbarb}{get.Rul.lbarb}};
\code{\LinkA{get\_lb}{get.Rul.lb}}; \code{\LinkA{get\_lp}{get.Rul.lp}};
\code{\LinkA{get\_tb}{get.Rul.tb}}; \code{\LinkA{get\_tm\_s}{get.Rul.tm.Rul.s}};
\code{\LinkA{get\_tp}{get.Rul.tp}};
\code{\LinkA{initial\_scaled\_reserve}{initial.Rul.scaled.Rul.reserve}};
\code{\LinkA{reprod\_rate}{reprod.Rul.rate}}
\end{SeeAlso}
%
\begin{Examples}
\begin{ExampleCode}
dget_lbarb2(10^(-6), 0, c(lbarb = 0.003, xb = 10/11, xb3 = (10/11)^(1/3), k = 1))
\end{ExampleCode}
\end{Examples}
\inputencoding{utf8}
\HeaderA{dget\_l\_ISO}{Computes derivative d l/d vH}{dget.Rul.l.Rul.ISO}
%
\begin{Description}\relax
Obtains the derivative d l/d vH from g, k, lT, f, sM.
\end{Description}
%
\begin{Usage}
\begin{verbatim}
dget_l_ISO(vH, l, pars)
\end{verbatim}
\end{Usage}
%
\begin{Arguments}
\begin{ldescription}
\item[\code{vH}] scaled maturity volume

\item[\code{l}] scaled length

\item[\code{pars}] data.frame with g, k, lT, f, sM
\end{ldescription}
\end{Arguments}
%
\begin{Value}
scalar with derivative value d l/ d vH
\end{Value}
\inputencoding{utf8}
\HeaderA{estim\_options}{Sets options for estim\_pars}{estim.Rul.options}
%
\begin{Description}\relax
Sets options for estimation one by one
\end{Description}
%
\begin{Usage}
\begin{verbatim}
estim_options(key = "inexistent", val = "")
\end{verbatim}
\end{Usage}
%
\begin{Arguments}
\begin{ldescription}
\item[\code{key}] string with option to set

\item[\code{val}] value of the option
\end{ldescription}
\end{Arguments}
%
\begin{Details}\relax
no input: print values to screen

one input:

* "default": sets options at default values
* any other key (see below): print value to screen

two inputs:

* "filter": 1 - use filter (default); 0 - do not;
* "pars\_init\_method":
0 - get initial estimates from automatized computation (default)
1 - read initial estimates from .mat file (for continuation)
2 - read initial estimates from pars\_init file
* "pseudodata\_pets":
0 - put pseudodata together with data (default)
1 - put it apart (only for multispecies estimation)
* "results\_output":
0 - prints results to screen (default)
1 - prints results to screen, saves to .mat file
2 - saves data to .mat file and graphs to .png files
(prints results to screen using a customized results file when it exists)
* "method": "nm" - use Nelder-Mead method; "no" - do not estimate;

for other options see corresponding options file of the method (e.g. nmregr\_options)
\end{Details}
%
\begin{SeeAlso}\relax
Other regression functions: \code{\LinkA{nmregr\_options}{nmregr.Rul.options}}
\end{SeeAlso}
%
\begin{Examples}
\begin{ExampleCode}
estim_options("default")
\end{ExampleCode}
\end{Examples}
\inputencoding{utf8}
\HeaderA{estim\_pars}{Estimates parameters}{estim.Rul.pars}
%
\begin{Description}\relax
Runs the entire estimation procedures: gets the parameters, gets the data, initiates the estimation procedure and sends the results for handling
\end{Description}
%
\begin{Usage}
\begin{verbatim}
estim_pars()
\end{verbatim}
\end{Usage}
%
\begin{SeeAlso}\relax
Other add-my-pet functions: \code{\LinkA{mydata\_pets}{mydata.Rul.pets}};
\code{\LinkA{petregr\_f}{petregr.Rul.f}}; \code{\LinkA{predict\_pets}{predict.Rul.pets}};
\code{\LinkA{printpar}{printpar}}; \code{\LinkA{printprd}{printprd}};
\code{\LinkA{results\_pets}{results.Rul.pets}}
\end{SeeAlso}
%
\begin{Examples}
\begin{ExampleCode}
estim_pars()
\end{ExampleCode}
\end{Examples}
\inputencoding{utf8}
\HeaderA{fieldnmnst}{Creates a list of field names of a structure}{fieldnmnst}
%
\begin{Description}\relax
Creates a list of field names of a structure
\end{Description}
%
\begin{Usage}
\begin{verbatim}
fieldnmnst(st)
\end{verbatim}
\end{Usage}
%
\begin{Arguments}
\begin{ldescription}
\item[\code{st}] data frame with fields
\end{ldescription}
\end{Arguments}
%
\begin{Value}
list of list of strings with fields including the field name in str
\end{Value}
%
\begin{SeeAlso}\relax
Other add-my-pet auxiliary functions: \code{\LinkA{addchem}{addchem}};
\code{\LinkA{addpseudodata}{addpseudodata}}; \code{\LinkA{fieldnm\_wtxt}{fieldnm.Rul.wtxt}};
\code{\LinkA{get\_d\_V}{get.Rul.d.Rul.V}}; \code{\LinkA{parscomp}{parscomp}};
\code{\LinkA{predict\_pseudodata}{predict.Rul.pseudodata}};
\code{\LinkA{print\_filterflag}{print.Rul.filterflag}};
\code{\LinkA{rmpseudodata}{rmpseudodata}}; \code{\LinkA{setweights}{setweights}}
\end{SeeAlso}
%
\begin{Examples}
\begin{ExampleCode}
nst <- fieldnmnst(st)
\end{ExampleCode}
\end{Examples}
\inputencoding{utf8}
\HeaderA{fieldnm\_wtxt}{Searches for fields with a given name in a multilevel structure}{fieldnm.Rul.wtxt}
%
\begin{Description}\relax
Creates a list of field names of a data frame with str
\end{Description}
%
\begin{Usage}
\begin{verbatim}
fieldnm_wtxt(data = list(), str = "")
\end{verbatim}
\end{Usage}
%
\begin{Arguments}
\begin{ldescription}
\item[\code{data}] data frame with fields

\item[\code{str}] string with field name
\end{ldescription}
\end{Arguments}
%
\begin{Value}
list of strings with fields including the field name in str
\end{Value}
%
\begin{SeeAlso}\relax
Other add-my-pet auxiliary functions: \code{\LinkA{addchem}{addchem}};
\code{\LinkA{addpseudodata}{addpseudodata}}; \code{\LinkA{fieldnmnst}{fieldnmnst}};
\code{\LinkA{get\_d\_V}{get.Rul.d.Rul.V}}; \code{\LinkA{parscomp}{parscomp}};
\code{\LinkA{predict\_pseudodata}{predict.Rul.pseudodata}};
\code{\LinkA{print\_filterflag}{print.Rul.filterflag}};
\code{\LinkA{rmpseudodata}{rmpseudodata}}; \code{\LinkA{setweights}{setweights}}
\end{SeeAlso}
%
\begin{Examples}
\begin{ExampleCode}
nmpsd <- fieldnm_wtxt(data, "psd")
\end{ExampleCode}
\end{Examples}
\inputencoding{utf8}
\HeaderA{filter\_std}{Filters for allowed parameters of standard DEB model without acceleration}{filter.Rul.std}
%
\begin{Description}\relax
Checks if parameter values are in the allowable part of the parameter space of
standard DEB model without acceleration. Meant to be run in the estimation procedure
\end{Description}
%
\begin{Usage}
\begin{verbatim}
filter_std(par)
\end{verbatim}
\end{Usage}
%
\begin{Arguments}
\begin{ldescription}
\item[\code{par}] data frame with parameter values
\end{ldescription}
\end{Arguments}
%
\begin{Details}\relax
The flag is an indicator of reason for not passing the filter and it means
0: parameters pass the filter
1: some parameter is negative
2: some kappa is larger than 1
3: growth efficiency is larger than 1
4: maturity levels do not increase during life cycle
5: puberty cannot be reached
\end{Details}
%
\begin{Value}
list with filter and flag
\end{Value}
%
\begin{SeeAlso}\relax
Other filter functions: \code{\LinkA{reach\_birth}{reach.Rul.birth}};
\code{\LinkA{warning\_std}{warning.Rul.std}}
\end{SeeAlso}
%
\begin{Examples}
\begin{ExampleCode}
filter_std(par)
\end{ExampleCode}
\end{Examples}
\inputencoding{utf8}
\HeaderA{fnget\_lbarb2}{Computes f using the ode solver for delta(x), for finding lbarb}{fnget.Rul.lbarb2}
%
\begin{Description}\relax
Computes f using the ode solver for delta(x), for finding lbarb.
\end{Description}
%
\begin{Usage}
\begin{verbatim}
fnget_lbarb2(lbarb, pars)
\end{verbatim}
\end{Usage}
%
\begin{Arguments}
\begin{ldescription}
\item[\code{lbarb}] scalar with scaled length at birth (lbarb = lb/ g)

\item[\code{pars}] data.frame with lbarb, xb, xb3 (xb\textasciicircum{}1/3), k
\end{ldescription}
\end{Arguments}
%
\begin{Value}
scalar with function f which when zero indicates lbarb
\end{Value}
%
\begin{SeeAlso}\relax
Other scaled get functions: \code{\LinkA{dget\_lbarb2}{dget.Rul.lbarb2}};
\code{\LinkA{get\_lbarb2}{get.Rul.lbarb2}}; \code{\LinkA{get\_lbarb}{get.Rul.lbarb}};
\code{\LinkA{get\_lb}{get.Rul.lb}}; \code{\LinkA{get\_lp}{get.Rul.lp}};
\code{\LinkA{get\_tb}{get.Rul.tb}}; \code{\LinkA{get\_tm\_s}{get.Rul.tm.Rul.s}};
\code{\LinkA{get\_tp}{get.Rul.tp}};
\code{\LinkA{initial\_scaled\_reserve}{initial.Rul.scaled.Rul.reserve}};
\code{\LinkA{reprod\_rate}{reprod.Rul.rate}}
\end{SeeAlso}
%
\begin{Examples}
\begin{ExampleCode}
fnget_lbarb2(0.03, c(xb = 10/11, xb3 = (10/11)^(1/3), vbarHb = 0.001, k = 1))
\end{ExampleCode}
\end{Examples}
\inputencoding{utf8}
\HeaderA{get\_d\_V}{Sets chemical parameters and text for units and labels}{get.Rul.d.Rul.V}
%
\begin{Description}\relax
Sets chemical parameters and text for units and labels
\end{Description}
%
\begin{Usage}
\begin{verbatim}
get_d_V(phylum, class)
\end{verbatim}
\end{Usage}
%
\begin{Arguments}
\begin{ldescription}
\item[\code{phylum}] string with species phylum

\item[\code{class}] string with species class
\end{ldescription}
\end{Arguments}
%
\begin{Details}\relax
Calls get\_d\_V to set specific density of structure. For a specific density of wet mass of 1 g/cm\textasciicircum{}3,
a specific density of d\_E = d\_V = 0.1 g/cm\textasciicircum{}3 means a dry-over-wet weight ratio of 0.1
\end{Details}
%
\begin{Value}
list with d\_V and info
\end{Value}
%
\begin{SeeAlso}\relax
Other add-my-pet auxiliary functions: \code{\LinkA{addchem}{addchem}};
\code{\LinkA{addpseudodata}{addpseudodata}}; \code{\LinkA{fieldnm\_wtxt}{fieldnm.Rul.wtxt}};
\code{\LinkA{fieldnmnst}{fieldnmnst}}; \code{\LinkA{parscomp}{parscomp}};
\code{\LinkA{predict\_pseudodata}{predict.Rul.pseudodata}};
\code{\LinkA{print\_filterflag}{print.Rul.filterflag}};
\code{\LinkA{rmpseudodata}{rmpseudodata}}; \code{\LinkA{setweights}{setweights}}
\end{SeeAlso}
%
\begin{Examples}
\begin{ExampleCode}
get_d_V("Chordata", "Mammalia")
\end{ExampleCode}
\end{Examples}
\inputencoding{utf8}
\HeaderA{get\_lb}{Computes scaled length at birth}{get.Rul.lb}
%
\begin{Description}\relax
Obtains scaled length at birth, given the scaled reserve density at birth.
\end{Description}
%
\begin{Usage}
\begin{verbatim}
get_lb(pars, eb = 1, lb0 = as.numeric(pars[3]^(1/3)))
\end{verbatim}
\end{Usage}
%
\begin{Arguments}
\begin{ldescription}
\item[\code{pars}] 3-vector with parameters: g, k, v\_H\textasciicircum{}b

\item[\code{eb}] optional scalar with scaled reserve density at birth (default eb = 1)

\item[\code{lb0}] optional scalar with initial estimate for scaled length at birth (default lb0: lb for k = 1)
\end{ldescription}
\end{Arguments}
%
\begin{Details}\relax
The theory behind get\_lb, get\_tb and get\_ue0 is discussed in http://www.bio.vu.nl/thb/research/bib/Kooy2009b.html.
Solves \eqn{y(x_b) = y_b}{}  for \eqn{l_b}{} with explicit solution for \eqn{y(x)}{}
\deqn{y(x) = \frac{x e_H}{1-kap} = x g \frac{u_H}{l^3}}{}
and \eqn{y_b = x_b g u_H^b/ ((1-kap)l_b^3)}{}
\deqn{\frac{d}{dx} y = r(x) - y s(x)}{}
with solution \eqn{y(x) = v(x) \int_0^x r(x')/ v(x') dx'}{}
and \eqn{v(x) = exp(- \int_0^x s(x') dx')}{}.
A Newton Raphson scheme is used with Euler integration, starting from an optional initial value.
Shooting method: get\_lb2.
In case of no convergence, get\_lb2 is run automatically as backup.
Consider the application of get\_lb\_foetus for an alternative initial value.
\end{Details}
%
\begin{Value}
scalar with scaled length at birth (lb) and indicator equals 1 if successful convergence, 0 otherwise (info)
\end{Value}
%
\begin{SeeAlso}\relax
Other scaled get functions: \code{\LinkA{dget\_lbarb2}{dget.Rul.lbarb2}};
\code{\LinkA{fnget\_lbarb2}{fnget.Rul.lbarb2}}; \code{\LinkA{get\_lbarb2}{get.Rul.lbarb2}};
\code{\LinkA{get\_lbarb}{get.Rul.lbarb}}; \code{\LinkA{get\_lp}{get.Rul.lp}};
\code{\LinkA{get\_tb}{get.Rul.tb}}; \code{\LinkA{get\_tm\_s}{get.Rul.tm.Rul.s}};
\code{\LinkA{get\_tp}{get.Rul.tp}};
\code{\LinkA{initial\_scaled\_reserve}{initial.Rul.scaled.Rul.reserve}};
\code{\LinkA{reprod\_rate}{reprod.Rul.rate}}
\end{SeeAlso}
%
\begin{Examples}
\begin{ExampleCode}
get_lb(c(g = 10, k = 1, vHb = 0.5), 1)
\end{ExampleCode}
\end{Examples}
\inputencoding{utf8}
\HeaderA{get\_lbarb}{Computes scaled length at birth lbarb}{get.Rul.lbarb}
%
\begin{Description}\relax
Obtains scaled length at birth, given the scaled reserve density at birth.
\end{Description}
%
\begin{Usage}
\begin{verbatim}
get_lbarb(pars, eb = 1, lbarb0 = NA)
\end{verbatim}
\end{Usage}
%
\begin{Arguments}
\begin{ldescription}
\item[\code{pars}] 3-vector with parameters: g, k, vbar\_H\textasciicircum{}b

\item[\code{eb}] optional scalar with scaled reserve density at birth (default eb = 1)

\item[\code{lbarb0}] optional scalar with initial estimate for scaled length at birth (default lbarb0: lbarb for k = 1)
\end{ldescription}
\end{Arguments}
%
\begin{Value}
scalar with scaled length at birth (lbarb) and indicator equals 1 if successful, 0 otherwise (info)
\end{Value}
%
\begin{SeeAlso}\relax
Other scaled get functions: \code{\LinkA{dget\_lbarb2}{dget.Rul.lbarb2}};
\code{\LinkA{fnget\_lbarb2}{fnget.Rul.lbarb2}}; \code{\LinkA{get\_lbarb2}{get.Rul.lbarb2}};
\code{\LinkA{get\_lb}{get.Rul.lb}}; \code{\LinkA{get\_lp}{get.Rul.lp}};
\code{\LinkA{get\_tb}{get.Rul.tb}}; \code{\LinkA{get\_tm\_s}{get.Rul.tm.Rul.s}};
\code{\LinkA{get\_tp}{get.Rul.tp}};
\code{\LinkA{initial\_scaled\_reserve}{initial.Rul.scaled.Rul.reserve}};
\code{\LinkA{reprod\_rate}{reprod.Rul.rate}}
\end{SeeAlso}
%
\begin{Examples}
\begin{ExampleCode}
get_lbarb(c(g = 10, k = 1, vbarHb = 0.0005), 1)
\end{ExampleCode}
\end{Examples}
\inputencoding{utf8}
\HeaderA{get\_lbarb2}{Computes initial scaled reserve}{get.Rul.lbarb2}
%
\begin{Description}\relax
Obtains scaled length at birth, given the scaled reserve density at birth. Like get\_lbarb, but uses a shooting method in 1 variable.
\end{Description}
%
\begin{Usage}
\begin{verbatim}
get_lbarb2(pars, eb = NA)
\end{verbatim}
\end{Usage}
%
\begin{Arguments}
\begin{ldescription}
\item[\code{pars}] 3-vector with parameters: g, k, vbar\_H\textasciicircum{}b

\item[\code{eb}] optional scalar with scaled reserve density at birth (default eb = 1)
\end{ldescription}
\end{Arguments}
%
\begin{Value}
scalar with scaled length at birth (lbarb) and indicator equals 1 if successful, 0 otherwise (info)
\end{Value}
%
\begin{SeeAlso}\relax
Other scaled get functions: \code{\LinkA{dget\_lbarb2}{dget.Rul.lbarb2}};
\code{\LinkA{fnget\_lbarb2}{fnget.Rul.lbarb2}}; \code{\LinkA{get\_lbarb}{get.Rul.lbarb}};
\code{\LinkA{get\_lb}{get.Rul.lb}}; \code{\LinkA{get\_lp}{get.Rul.lp}};
\code{\LinkA{get\_tb}{get.Rul.tb}}; \code{\LinkA{get\_tm\_s}{get.Rul.tm.Rul.s}};
\code{\LinkA{get\_tp}{get.Rul.tp}};
\code{\LinkA{initial\_scaled\_reserve}{initial.Rul.scaled.Rul.reserve}};
\code{\LinkA{reprod\_rate}{reprod.Rul.rate}}
\end{SeeAlso}
%
\begin{Examples}
\begin{ExampleCode}
get_lbarb2(c(g = 10, k = 1, vbarHb = 0.01), 1)
\end{ExampleCode}
\end{Examples}
\inputencoding{utf8}
\HeaderA{get\_lp}{Computes scaled length at puberty}{get.Rul.lp}
%
\begin{Description}\relax
Obtains scaled length at pubertyat constant food density.
\end{Description}
%
\begin{Usage}
\begin{verbatim}
get_lp(pars, f = 1, lb0 = NA)
\end{verbatim}
\end{Usage}
%
\begin{Arguments}
\begin{ldescription}
\item[\code{pars}] 5-vector with parameters: g, k, l\_T, v\_H\textasciicircum{}b, v\_H\textasciicircum{}p

\item[\code{lb0}] optional scalar with initial estimate for scaled length at birth (default lb0: lb for k = 1)

\item[\code{eb}] optional scalar with scaled reserve density at birth (default eb = 1)
\end{ldescription}
\end{Arguments}
%
\begin{Details}\relax
If scaled length at birth (second input) is not specified, it is computed (using automatic initial estimate).
If it is specified, however, is it just copied to the (second) output. Food density is assumed to be constant.
\end{Details}
%
\begin{Value}
scaled length at puberty (lp), scaler length at birth (lb)
and indicator equals 1 if successful convergence, 0 otherwise (info)
\end{Value}
%
\begin{SeeAlso}\relax
Other scaled get functions: \code{\LinkA{dget\_lbarb2}{dget.Rul.lbarb2}};
\code{\LinkA{fnget\_lbarb2}{fnget.Rul.lbarb2}}; \code{\LinkA{get\_lbarb2}{get.Rul.lbarb2}};
\code{\LinkA{get\_lbarb}{get.Rul.lbarb}}; \code{\LinkA{get\_lb}{get.Rul.lb}};
\code{\LinkA{get\_tb}{get.Rul.tb}}; \code{\LinkA{get\_tm\_s}{get.Rul.tm.Rul.s}};
\code{\LinkA{get\_tp}{get.Rul.tp}};
\code{\LinkA{initial\_scaled\_reserve}{initial.Rul.scaled.Rul.reserve}};
\code{\LinkA{reprod\_rate}{reprod.Rul.rate}}
\end{SeeAlso}
%
\begin{Examples}
\begin{ExampleCode}
get_lp(c(g = 10, k = 1, lT = 0, vHb = 0.5, vHp = 10), 1)
\end{ExampleCode}
\end{Examples}
\inputencoding{utf8}
\HeaderA{get\_tb}{Gets scaled age at birth}{get.Rul.tb}
%
\begin{Description}\relax
Obtains scaled age at birth, given the scaled reserve density at birth.
\end{Description}
%
\begin{Usage}
\begin{verbatim}
get_tb(pars, eb = 1, lb = NA)
\end{verbatim}
\end{Usage}
%
\begin{Arguments}
\begin{ldescription}
\item[\code{pars}] 3-vector with parameters: g, k, v\_H\textasciicircum{}b

\item[\code{eb}] optional scalar with scaled reserve density at birth (default eb = 1)

\item[\code{lb0}] optional scalar with initial estimate for scaled length at birth (default lb0: lb for k = 1)
\end{ldescription}
\end{Arguments}
%
\begin{Details}\relax
Multiply the result with the somatic maintenance rate coefficient to arrive at age at puberty.
\end{Details}
%
\begin{Value}
list with scaled age at birth tau\_b = a\_b k\_M (tb), scaled length at birth (lb)
and indicator equals 1 if successful convergence, 0 otherwise (info)
\end{Value}
%
\begin{SeeAlso}\relax
Other scaled get functions: \code{\LinkA{dget\_lbarb2}{dget.Rul.lbarb2}};
\code{\LinkA{fnget\_lbarb2}{fnget.Rul.lbarb2}}; \code{\LinkA{get\_lbarb2}{get.Rul.lbarb2}};
\code{\LinkA{get\_lbarb}{get.Rul.lbarb}}; \code{\LinkA{get\_lb}{get.Rul.lb}};
\code{\LinkA{get\_lp}{get.Rul.lp}}; \code{\LinkA{get\_tm\_s}{get.Rul.tm.Rul.s}};
\code{\LinkA{get\_tp}{get.Rul.tp}};
\code{\LinkA{initial\_scaled\_reserve}{initial.Rul.scaled.Rul.reserve}};
\code{\LinkA{reprod\_rate}{reprod.Rul.rate}}
\end{SeeAlso}
%
\begin{Examples}
\begin{ExampleCode}
get_tb(c(g = 10, k = 1, vHb = 0.5), 1)
\end{ExampleCode}
\end{Examples}
\inputencoding{utf8}
\HeaderA{get\_tm\_s}{Obtains scaled mean age at death for short growth periods}{get.Rul.tm.Rul.s}
%
\begin{Description}\relax
Obtains scaled mean age at death assuming a short growth period relative to the life span
\end{Description}
%
\begin{Usage}
\begin{verbatim}
get_tm_s(pars, f = 1, lb = NA, lp = NA)
\end{verbatim}
\end{Usage}
%
\begin{Arguments}
\begin{ldescription}
\item[\code{pars}] 4 or 7-vector with parameters: [g lT ha sG] or [g k lT vHb vHp ha SG]

\item[\code{f}] optional scalar with scaled reserve density at birth (default f = 1)

\item[\code{lb}] optional scalar with scaled length at birth (default: lb is obtained from get\_lb)

\item[\code{lp}] optional scalar with scaled length at puberty
\end{ldescription}
\end{Arguments}
%
\begin{Details}\relax
Divide the result by the somatic maintenance rate coefficient to arrive at the mean age at death.
The variant get\_tm\_foetus does the same in case of foetal development.
If the input parameter vector has only 4 elements (for [g, lT, ha/ kM2, sG]),
it skips the calulation of the survival probability at birth and puberty.
\end{Details}
%
\begin{Value}
list with  scalar with scaled mean life span (tm),
scalar with survival probability at birth (if length p = 7) (Sb),
scalar with survival prabability at puberty (if length p = 7) (Sp)
and indicator equals 1 if successful convergence, 0 otherwise (info)
\end{Value}
%
\begin{SeeAlso}\relax
Other scaled get functions: \code{\LinkA{dget\_lbarb2}{dget.Rul.lbarb2}};
\code{\LinkA{fnget\_lbarb2}{fnget.Rul.lbarb2}}; \code{\LinkA{get\_lbarb2}{get.Rul.lbarb2}};
\code{\LinkA{get\_lbarb}{get.Rul.lbarb}}; \code{\LinkA{get\_lb}{get.Rul.lb}};
\code{\LinkA{get\_lp}{get.Rul.lp}}; \code{\LinkA{get\_tb}{get.Rul.tb}};
\code{\LinkA{get\_tp}{get.Rul.tp}};
\code{\LinkA{initial\_scaled\_reserve}{initial.Rul.scaled.Rul.reserve}};
\code{\LinkA{reprod\_rate}{reprod.Rul.rate}}
\end{SeeAlso}
\inputencoding{utf8}
\HeaderA{get\_tp}{Gets scaled age at puberty}{get.Rul.tp}
%
\begin{Description}\relax
Obtains scaled age at puberty.
\end{Description}
%
\begin{Usage}
\begin{verbatim}
get_tp(pars, f = 1, lb0 = as.numeric(pars[4]^(1/3)))
\end{verbatim}
\end{Usage}
%
\begin{Arguments}
\begin{ldescription}
\item[\code{pars}] 5-vector with parameters: g, k, l\_T, v\_H\textasciicircum{}b, v\_H\textasciicircum{}p

\item[\code{f}] optional scalar with functional response (default f = 1)

\item[\code{lb0}] optional scalar with scaled length at birth (default lb0: lb for k = 1)
\end{ldescription}
\end{Arguments}
%
\begin{Details}\relax
Food density is assumed to be constant. Multiply the result with the
somatic maintenance rate coefficient to arrive at age at puberty.
\end{Details}
%
\begin{Value}
list with scaled age at puberty tau\_p = a\_p k\_M (tp),
scaled age at birth tau\_b = a\_b k\_M (tb),
scaled length at puberty (lp), scaled length at birth (lb)
and indicator equals 1 if successful convergence, 0 otherwise (info)
\end{Value}
%
\begin{SeeAlso}\relax
Other scaled get functions: \code{\LinkA{dget\_lbarb2}{dget.Rul.lbarb2}};
\code{\LinkA{fnget\_lbarb2}{fnget.Rul.lbarb2}}; \code{\LinkA{get\_lbarb2}{get.Rul.lbarb2}};
\code{\LinkA{get\_lbarb}{get.Rul.lbarb}}; \code{\LinkA{get\_lb}{get.Rul.lb}};
\code{\LinkA{get\_lp}{get.Rul.lp}}; \code{\LinkA{get\_tb}{get.Rul.tb}};
\code{\LinkA{get\_tm\_s}{get.Rul.tm.Rul.s}};
\code{\LinkA{initial\_scaled\_reserve}{initial.Rul.scaled.Rul.reserve}};
\code{\LinkA{reprod\_rate}{reprod.Rul.rate}}
\end{SeeAlso}
%
\begin{Examples}
\begin{ExampleCode}
get_tp(c(g = 10, k = 1, lT = 0, vHb = 0.5, vHp = 10), 1)
\end{ExampleCode}
\end{Examples}
\inputencoding{utf8}
\HeaderA{get\_ubarE0}{Computes initial scaled reserve density at birth}{get.Rul.ubarE0}
%
\begin{Description}\relax
Obtains the initial scaled reserve given the scaled reserve density at birth.
Function get\_ue0 does so for eggs, get\_ue0\_foetus for foetuses.
Specification of length at birth as third input by-passes its computation,
so if you want to specify an initial value for this quantity, you should use get\_lb directly.
\end{Description}
%
\begin{Usage}
\begin{verbatim}
get_ubarE0(g = NA, k = NA, vbarHb = NA, eb = 1, lbarb = NA)
\end{verbatim}
\end{Usage}
%
\begin{Arguments}
\begin{ldescription}
\item[\code{g}] energy investment ratio

\item[\code{k}] maintenance ratio

\item[\code{vbarHb}] rescaled maturity volume at birth

\item[\code{eb}] optional scalar with scaled reserbe density at birth

\item[\code{lbarb}] optional scalar with scaled length at birth
\end{ldescription}
\end{Arguments}
%
\begin{Value}
scalar with particular incomple beta function
\end{Value}
%
\begin{SeeAlso}\relax
Other get functions: \code{\LinkA{get\_ue0}{get.Rul.ue0}}
\end{SeeAlso}
%
\begin{Examples}
\begin{ExampleCode}
get_ubarE0(g = 10, lbarb = 0.01)
get_ubarE0(g = 10, k = 0.7, vbarHb = 5e-4)
\end{ExampleCode}
\end{Examples}
\inputencoding{utf8}
\HeaderA{get\_ue0}{Computes initial scaled reserve}{get.Rul.ue0}
%
\begin{Description}\relax
Obtains the initial scaled reserve given the scaled reserve density at birth.
Function get\_ue0 does so for eggs, get\_ue0\_foetus for foetuses.
Specification of length at birth as third input by-passes its computation,
so if you want to specify an initial value for this quantity, you should use get\_lb directly.
\end{Description}
%
\begin{Usage}
\begin{verbatim}
get_ue0(pars, eb = 1, lb0 = NA)
\end{verbatim}
\end{Usage}
%
\begin{Arguments}
\begin{ldescription}
\item[\code{pars}] 1 or 3 -vector with parameters g, k\_J/ k\_M, v\_H\textasciicircum{}b, see get\_lb

\item[\code{eb}] optional scalar with scaled reserbe density at birth (default: eb = 1)

\item[\code{lb0}] optional scalar with scaled length at birth (default: lb is optained from get\_lb)
\end{ldescription}
\end{Arguments}
%
\begin{Value}
uE0 scalar with scaled reserve at t=0: \$U\_E\textasciicircum{}0 g\textasciicircum{}2 k\_M\textasciicircum{}3/ v\textasciicircum{}2\$ with \$U\_E\textasciicircum{}0 = M\_E\textasciicircum{}0/ \{J\_EAm\}\$, lb scalar with scaled length at birth and info indicator equals 1 if successful, 0 otherwise
\end{Value}
%
\begin{SeeAlso}\relax
Other get functions: \code{\LinkA{get\_ubarE0}{get.Rul.ubarE0}}
\end{SeeAlso}
%
\begin{Examples}
\begin{ExampleCode}
get_ue0(pars = c(0.42, 1, 0.066), eb = 1, lb0 = 0.4042)
\end{ExampleCode}
\end{Examples}
\inputencoding{utf8}
\HeaderA{initial\_scaled\_reserve}{Gets initial scaled reserve}{initial.Rul.scaled.Rul.reserve}
%
\begin{Description}\relax
Gets initial scaled reserve.
\end{Description}
%
\begin{Usage}
\begin{verbatim}
initial_scaled_reserve(f, pars, Lb0 = NA)
\end{verbatim}
\end{Usage}
%
\begin{Arguments}
\begin{ldescription}
\item[\code{f}] n-vector with scaled functional responses

\item[\code{pars}] 5-vector with parameters: VHb, g, kJ, kM, v

\item[\code{Lb0}] optional n-vector with lengths at birth
\end{ldescription}
\end{Arguments}
%
\begin{Value}
n-vector with initial scaled reserve: M\_E\textasciicircum{}0/ J\_EAm (U0), n-vector with length at birth (Lb) and n-vector with 1's if successful, 0's otherwise (info)
\end{Value}
%
\begin{SeeAlso}\relax
Other scaled get functions: \code{\LinkA{dget\_lbarb2}{dget.Rul.lbarb2}};
\code{\LinkA{fnget\_lbarb2}{fnget.Rul.lbarb2}}; \code{\LinkA{get\_lbarb2}{get.Rul.lbarb2}};
\code{\LinkA{get\_lbarb}{get.Rul.lbarb}}; \code{\LinkA{get\_lb}{get.Rul.lb}};
\code{\LinkA{get\_lp}{get.Rul.lp}}; \code{\LinkA{get\_tb}{get.Rul.tb}};
\code{\LinkA{get\_tm\_s}{get.Rul.tm.Rul.s}}; \code{\LinkA{get\_tp}{get.Rul.tp}};
\code{\LinkA{reprod\_rate}{reprod.Rul.rate}}
\end{SeeAlso}
%
\begin{Examples}
\begin{ExampleCode}
initial_scaled_reserve(f = c(1, 0.9), pars = c(VHb = .8, g = .42, kJ = 1.7, kM = 1.7, v = 3.24))
\end{ExampleCode}
\end{Examples}
\inputencoding{utf8}
\HeaderA{K2C}{Conversion of Kelvin to Celsius}{K2C}
%
\begin{Description}\relax
Converts temperature in Kelvin to degrees Celsius
\end{Description}
%
\begin{Usage}
\begin{verbatim}
K2C(K)
\end{verbatim}
\end{Usage}
%
\begin{Arguments}
\begin{ldescription}
\item[\code{K}] numeric temperature in degrees Kelvin
\end{ldescription}
\end{Arguments}
%
\begin{Value}
temperature in Kelvin
\end{Value}
%
\begin{SeeAlso}\relax
Other miscellaneous functions: \code{\LinkA{C2K}{C2K}};
\code{\LinkA{beta0}{beta0}}
\end{SeeAlso}
%
\begin{Examples}
\begin{ExampleCode}
K2C(293.15)
\end{ExampleCode}
\end{Examples}
\inputencoding{utf8}
\HeaderA{mre\_st}{Computes mean relative error}{mre.Rul.st}
%
\begin{Description}\relax
Computes relative errors and mean relative error for using data and predictions
\end{Description}
%
\begin{Usage}
\begin{verbatim}
mre_st(func, par, data, auxData, weights)
\end{verbatim}
\end{Usage}
%
\begin{Arguments}
\begin{ldescription}
\item[\code{func}] string with predict file name

\item[\code{par}] data frame with parameter values

\item[\code{data}] data frame with data values

\item[\code{auxData}] data frame with auxiliary data values

\item[\code{weights}] data frame with values of weights
\end{ldescription}
\end{Arguments}
%
\begin{Examples}
\begin{ExampleCode}
results_pets(par, metaPar, txtPar, data, auxData, metaData, txtData, weights)
\end{ExampleCode}
\end{Examples}
\inputencoding{utf8}
\HeaderA{mydata\_my\_pet}{Sets referenced data}{mydata.Rul.my.Rul.pet}
%
\begin{Description}\relax
Sets data, pseudodata, metadata, auxdata, explanatory text, weights coefficients. Meant to be a template in add-my-pet
\end{Description}
%
\begin{Usage}
\begin{verbatim}
mydata_my_pet()
\end{verbatim}
\end{Usage}
%
\begin{Value}
list with data, auxData, metaData, txtData and weights
\end{Value}
%
\begin{SeeAlso}\relax
Other add-my-pet template functions: \code{\LinkA{pars\_init\_my\_pet}{pars.Rul.init.Rul.my.Rul.pet}};
\code{\LinkA{predict\_my\_pet}{predict.Rul.my.Rul.pet}}
\end{SeeAlso}
%
\begin{Examples}
\begin{ExampleCode}
mydata_my_pet()
\end{ExampleCode}
\end{Examples}
\inputencoding{utf8}
\HeaderA{mydata\_pets}{Concatenates mydata files for several species}{mydata.Rul.pets}
%
\begin{Description}\relax
Concatenates mydata files for several species
\end{Description}
%
\begin{Usage}
\begin{verbatim}
mydata_pets()
\end{verbatim}
\end{Usage}
%
\begin{Value}
structure with data, auxData, metaData, txtData and weights for several pets
\end{Value}
%
\begin{SeeAlso}\relax
Other add-my-pet functions: \code{\LinkA{estim\_pars}{estim.Rul.pars}};
\code{\LinkA{petregr\_f}{petregr.Rul.f}}; \code{\LinkA{predict\_pets}{predict.Rul.pets}};
\code{\LinkA{printpar}{printpar}}; \code{\LinkA{printprd}{printprd}};
\code{\LinkA{results\_pets}{results.Rul.pets}}
\end{SeeAlso}
%
\begin{Examples}
\begin{ExampleCode}
mydata_pets()
\end{ExampleCode}
\end{Examples}
\inputencoding{utf8}
\HeaderA{nmregr\_options}{Sets options for function nmregr}{nmregr.Rul.options}
%
\begin{Description}\relax
Sets options for estimation one by one
\end{Description}
%
\begin{Usage}
\begin{verbatim}
nmregr_options(key = "inexistent", val = "")
\end{verbatim}
\end{Usage}
%
\begin{Arguments}
\begin{ldescription}
\item[\code{key}] string with option to set

\item[\code{val}] value of the option
\end{ldescription}
\end{Arguments}
%
\begin{Details}\relax
no input: print values to screen

one input:

* "default": sets options at default values
* any other key (see below): print value to screen

two inputs:

* "report": 1 - to report steps to screen; 0 - not to;
* "max\_step\_number": maximum number of steps
* "max\_fun\_evals": maximum number of function evaluations
* "tol\_simplex": tolerance for how close the simplex points must be together to call them the same
* "tol\_tun": tolerance for how close the loss-function values must be together to call them the same
* "simplex\_size": fraction added (subtracted if negative) to the free parameters when building the simplex
\end{Details}
%
\begin{Value}
1 if input is valid key, 0 if input is unknown key
\end{Value}
%
\begin{SeeAlso}\relax
Other regression functions: \code{\LinkA{estim\_options}{estim.Rul.options}}
\end{SeeAlso}
%
\begin{Examples}
\begin{ExampleCode}
nmregr_options("default")
\end{ExampleCode}
\end{Examples}
\inputencoding{utf8}
\HeaderA{parscomp}{Computes compound parameters from primary parameters}{parscomp}
%
\begin{Description}\relax
Computes compound parameters from primary parameters that are frequently used
\end{Description}
%
\begin{Usage}
\begin{verbatim}
parscomp(par)
\end{verbatim}
\end{Usage}
%
\begin{Arguments}
\begin{ldescription}
\item[\code{par}] data frame with parameter values
\end{ldescription}
\end{Arguments}
%
\begin{Value}
list with compound parameters
\end{Value}
%
\begin{SeeAlso}\relax
Other add-my-pet auxiliary functions: \code{\LinkA{addchem}{addchem}};
\code{\LinkA{addpseudodata}{addpseudodata}}; \code{\LinkA{fieldnm\_wtxt}{fieldnm.Rul.wtxt}};
\code{\LinkA{fieldnmnst}{fieldnmnst}}; \code{\LinkA{get\_d\_V}{get.Rul.d.Rul.V}};
\code{\LinkA{predict\_pseudodata}{predict.Rul.pseudodata}};
\code{\LinkA{print\_filterflag}{print.Rul.filterflag}};
\code{\LinkA{rmpseudodata}{rmpseudodata}}; \code{\LinkA{setweights}{setweights}}
\end{SeeAlso}
%
\begin{Examples}
\begin{ExampleCode}
parscomp(par)
\end{ExampleCode}
\end{Examples}
\inputencoding{utf8}
\HeaderA{pars\_init\_my\_pet}{Sets (initial values for) parameters}{pars.Rul.init.Rul.my.Rul.pet}
%
\begin{Description}\relax
Sets (initial values for) parameters\$ Meant to be a template in add-my-pet
\end{Description}
%
\begin{Usage}
\begin{verbatim}
pars_init_my_pet(metaData)
\end{verbatim}
\end{Usage}
%
\begin{Arguments}
\begin{ldescription}
\item[\code{metaData}] data frame with info about this entry (needed for names of phylum and class to get d\_V)
\end{ldescription}
\end{Arguments}
%
\begin{Value}
list with par (with values of parameters), metaPar (with information on metaparameters) and txtPar (with information on parameters)
\end{Value}
%
\begin{SeeAlso}\relax
Other add-my-pet template functions: \code{\LinkA{mydata\_my\_pet}{mydata.Rul.my.Rul.pet}};
\code{\LinkA{predict\_my\_pet}{predict.Rul.my.Rul.pet}}
\end{SeeAlso}
%
\begin{Examples}
\begin{ExampleCode}
pars_init_my_pet(metaData)
\end{ExampleCode}
\end{Examples}
\inputencoding{utf8}
\HeaderA{petregr\_f}{Calculates least squares estimates using Nelder Mead's simplex method using a filter}{petregr.Rul.f}
%
\begin{Description}\relax
Calculates least squares estimates using Nelder Mead's simplex method using a filter
\end{Description}
%
\begin{Usage}
\begin{verbatim}
petregr_f(func, par, data, auxData, weights, filternm)
\end{verbatim}
\end{Usage}
%
\begin{Arguments}
\begin{ldescription}
\item[\code{func}] character string with name of user-defined function

\item[\code{par}] list with parameters

\item[\code{data}] list with data

\item[\code{auxData}] list with auxiliary data

\item[\code{weights}] list with weights

\item[\code{filternm}] character string with name of user-defined filter function
\end{ldescription}
\end{Arguments}
%
\begin{Value}
list with list with parameters resulting from estimation procedure (par)
and indicator 1 if convergence has been successful or 0 otherwise (info)
\end{Value}
%
\begin{SeeAlso}\relax
Other add-my-pet functions: \code{\LinkA{estim\_pars}{estim.Rul.pars}};
\code{\LinkA{mydata\_pets}{mydata.Rul.pets}}; \code{\LinkA{predict\_pets}{predict.Rul.pets}};
\code{\LinkA{printpar}{printpar}}; \code{\LinkA{printprd}{printprd}};
\code{\LinkA{results\_pets}{results.Rul.pets}}
\end{SeeAlso}
\inputencoding{utf8}
\HeaderA{predict\_my\_pet}{Obtains predictions, using parameters and data}{predict.Rul.my.Rul.pet}
%
\begin{Description}\relax
Obtains predictions, using parameters and data
\end{Description}
%
\begin{Usage}
\begin{verbatim}
predict_my_pet(par, data, auxData = list())
\end{verbatim}
\end{Usage}
%
\begin{Arguments}
\begin{ldescription}
\item[\code{par}] data frame with parameter values

\item[\code{data}] data frame with data values

\item[\code{auxData}] data frame with auxiliary data values
\end{ldescription}
\end{Arguments}
%
\begin{Value}
list with prdData (data frame with values of predictions) and info (indicator for customized filters)
\end{Value}
%
\begin{SeeAlso}\relax
Other add-my-pet template functions: \code{\LinkA{mydata\_my\_pet}{mydata.Rul.my.Rul.pet}};
\code{\LinkA{pars\_init\_my\_pet}{pars.Rul.init.Rul.my.Rul.pet}}
\end{SeeAlso}
%
\begin{Examples}
\begin{ExampleCode}
predict_my_pet(par, data, auxData)
\end{ExampleCode}
\end{Examples}
\inputencoding{utf8}
\HeaderA{predict\_pets}{Concatenates predict files for several species}{predict.Rul.pets}
%
\begin{Description}\relax
Concatenates predict files for several species
\end{Description}
%
\begin{Usage}
\begin{verbatim}
predict_pets(parGrp, data, auxData)
\end{verbatim}
\end{Usage}
%
\begin{Arguments}
\begin{ldescription}
\item[\code{parGrp}] data frame with parameter values of the group

\item[\code{data}] data frame with data values

\item[\code{auxData}] data frame with auxiliary data values
\end{ldescription}
\end{Arguments}
%
\begin{Value}
structure with prdData and prdInfo for several pets
\end{Value}
%
\begin{SeeAlso}\relax
Other add-my-pet functions: \code{\LinkA{estim\_pars}{estim.Rul.pars}};
\code{\LinkA{mydata\_pets}{mydata.Rul.pets}}; \code{\LinkA{petregr\_f}{petregr.Rul.f}};
\code{\LinkA{printpar}{printpar}}; \code{\LinkA{printprd}{printprd}};
\code{\LinkA{results\_pets}{results.Rul.pets}}
\end{SeeAlso}
\inputencoding{utf8}
\HeaderA{predict\_pseudodata}{Predicts pseudodata values}{predict.Rul.pseudodata}
%
\begin{Description}\relax
Adds pseudodata predictions into predictions structure
\end{Description}
%
\begin{Usage}
\begin{verbatim}
predict_pseudodata(par, data, prdData)
\end{verbatim}
\end{Usage}
%
\begin{Arguments}
\begin{ldescription}
\item[\code{par}] data frame with parameter values

\item[\code{data}] data frame with data values

\item[\code{prdData}] data frame with prediction values
\end{ldescription}
\end{Arguments}
%
\begin{Value}
structure with pseudodata predictions
\end{Value}
%
\begin{SeeAlso}\relax
Other add-my-pet auxiliary functions: \code{\LinkA{addchem}{addchem}};
\code{\LinkA{addpseudodata}{addpseudodata}}; \code{\LinkA{fieldnm\_wtxt}{fieldnm.Rul.wtxt}};
\code{\LinkA{fieldnmnst}{fieldnmnst}}; \code{\LinkA{get\_d\_V}{get.Rul.d.Rul.V}};
\code{\LinkA{parscomp}{parscomp}}; \code{\LinkA{print\_filterflag}{print.Rul.filterflag}};
\code{\LinkA{rmpseudodata}{rmpseudodata}}; \code{\LinkA{setweights}{setweights}}
\end{SeeAlso}
\inputencoding{utf8}
\HeaderA{printpar}{Prints parameters of a species to screen}{printpar}
%
\begin{Description}\relax
Prints parameters of a species to screen
\end{Description}
%
\begin{Usage}
\begin{verbatim}
printpar(par, txtPar)
\end{verbatim}
\end{Usage}
%
\begin{Arguments}
\begin{ldescription}
\item[\code{par}] list with parameter values

\item[\code{txtPar}] list with text info on parameters
\end{ldescription}
\end{Arguments}
%
\begin{SeeAlso}\relax
Other add-my-pet functions: \code{\LinkA{estim\_pars}{estim.Rul.pars}};
\code{\LinkA{mydata\_pets}{mydata.Rul.pets}}; \code{\LinkA{petregr\_f}{petregr.Rul.f}};
\code{\LinkA{predict\_pets}{predict.Rul.pets}}; \code{\LinkA{printprd}{printprd}};
\code{\LinkA{results\_pets}{results.Rul.pets}}
\end{SeeAlso}
\inputencoding{utf8}
\HeaderA{printprd}{Prints data of a species to screen}{printprd}
%
\begin{Description}\relax
Prints data of a species to screen
\end{Description}
%
\begin{Usage}
\begin{verbatim}
printprd(data, txtData, prdData, RE)
\end{verbatim}
\end{Usage}
%
\begin{Arguments}
\begin{ldescription}
\item[\code{data}] list with data values

\item[\code{txtData}] list with text info on data

\item[\code{prdData}] list with prediction values

\item[\code{RE}] list with relative errors
\end{ldescription}
\end{Arguments}
%
\begin{SeeAlso}\relax
Other add-my-pet functions: \code{\LinkA{estim\_pars}{estim.Rul.pars}};
\code{\LinkA{mydata\_pets}{mydata.Rul.pets}}; \code{\LinkA{petregr\_f}{petregr.Rul.f}};
\code{\LinkA{predict\_pets}{predict.Rul.pets}}; \code{\LinkA{printpar}{printpar}};
\code{\LinkA{results\_pets}{results.Rul.pets}}
\end{SeeAlso}
\inputencoding{utf8}
\HeaderA{print\_filterflag}{Prints an explanation of the filter flag onto the screen}{print.Rul.filterflag}
%
\begin{Description}\relax
Prints an explanation to the screen according to the flag produced by a filter.
Meant to be run in the estimation procedure for the seed parameter set
\end{Description}
%
\begin{Usage}
\begin{verbatim}
print_filterflag(flag)
\end{verbatim}
\end{Usage}
%
\begin{Arguments}
\begin{ldescription}
\item[\code{flag}] integer with code from filter
\end{ldescription}
\end{Arguments}
%
\begin{SeeAlso}\relax
Other add-my-pet auxiliary functions: \code{\LinkA{addchem}{addchem}};
\code{\LinkA{addpseudodata}{addpseudodata}}; \code{\LinkA{fieldnm\_wtxt}{fieldnm.Rul.wtxt}};
\code{\LinkA{fieldnmnst}{fieldnmnst}}; \code{\LinkA{get\_d\_V}{get.Rul.d.Rul.V}};
\code{\LinkA{parscomp}{parscomp}}; \code{\LinkA{predict\_pseudodata}{predict.Rul.pseudodata}};
\code{\LinkA{rmpseudodata}{rmpseudodata}}; \code{\LinkA{setweights}{setweights}}
\end{SeeAlso}
%
\begin{Examples}
\begin{ExampleCode}
print_filterflag(3)
\end{ExampleCode}
\end{Examples}
\inputencoding{utf8}
\HeaderA{reach\_birth}{Checks if parameters allow for reaching birth in the standard DEB model}{reach.Rul.birth}
%
\begin{Description}\relax
Checks if parameters allow for reaching birth in the standard DEB model
\end{Description}
%
\begin{Usage}
\begin{verbatim}
reach_birth(g, k, vHb, f = 1)
\end{verbatim}
\end{Usage}
%
\begin{Arguments}
\begin{ldescription}
\item[\code{g}] energy investment ratio

\item[\code{k}] ratio of maturity and somatic maintenance rate coeff

\item[\code{vHb}] scaled maturity volume at birth

\item[\code{f}] functional response (default 1)
\end{ldescription}
\end{Arguments}
%
\begin{Value}
info, indicator equals 1 if reaches birth, 0 otherwise
\end{Value}
%
\begin{SeeAlso}\relax
Other filter functions: \code{\LinkA{filter\_std}{filter.Rul.std}};
\code{\LinkA{warning\_std}{warning.Rul.std}}
\end{SeeAlso}
%
\begin{Examples}
\begin{ExampleCode}
reach_birth(g = 10, k = 1, vHb = 0.5)
\end{ExampleCode}
\end{Examples}
\inputencoding{utf8}
\HeaderA{reprod\_rate}{Gets reproduction rate}{reprod.Rul.rate}
%
\begin{Description}\relax
Calculates the reproduction rate in number of eggs per time for an individual of length L
and scaled reserve density f.
\end{Description}
%
\begin{Usage}
\begin{verbatim}
reprod_rate(L, f = 1, pars, Lf = NA)
\end{verbatim}
\end{Usage}
%
\begin{Arguments}
\begin{ldescription}
\item[\code{L}] n-vector with length

\item[\code{f}] scalar with functional response

\item[\code{pars}] 9-vector with parameters: kap, kapR, g, kJ, kM, LT, v, UHb, UHp

\item[\code{Lf}] optional scalar with length at birth (initial value only)
or optional 2-vector with length, L, and scaled functional response f0
for a juvenile that is now exposed to f, but previously at another f
\end{ldescription}
\end{Arguments}
%
\begin{Value}
list with n-vector with reproduction rates (R), scalar with scaled initial reserve (UE0),
scalar with (volumetric) length at birth (Lb), scalar with (volumetric) length at puberty (Lp) and
indicator with 1 for success, 0 otherwise (info)
\end{Value}
%
\begin{SeeAlso}\relax
Other scaled get functions: \code{\LinkA{dget\_lbarb2}{dget.Rul.lbarb2}};
\code{\LinkA{fnget\_lbarb2}{fnget.Rul.lbarb2}}; \code{\LinkA{get\_lbarb2}{get.Rul.lbarb2}};
\code{\LinkA{get\_lbarb}{get.Rul.lbarb}}; \code{\LinkA{get\_lb}{get.Rul.lb}};
\code{\LinkA{get\_lp}{get.Rul.lp}}; \code{\LinkA{get\_tb}{get.Rul.tb}};
\code{\LinkA{get\_tm\_s}{get.Rul.tm.Rul.s}}; \code{\LinkA{get\_tp}{get.Rul.tp}};
\code{\LinkA{initial\_scaled\_reserve}{initial.Rul.scaled.Rul.reserve}}
\end{SeeAlso}
\inputencoding{utf8}
\HeaderA{results\_pets}{Prints results of estimation}{results.Rul.pets}
%
\begin{Description}\relax
Prints the results of the esimation procedure in the screen, .mat file and makes figures of graphs
\end{Description}
%
\begin{Usage}
\begin{verbatim}
results_pets(par, metaPar, txtPar, data, auxData, metaData, txtData, weights)
\end{verbatim}
\end{Usage}
%
\begin{Arguments}
\begin{ldescription}
\item[\code{par}] data frame with parameter values

\item[\code{metaPar}] data frame with metainformation on models

\item[\code{txtPar}] data frame with information on parameters

\item[\code{data}] data frame with data values

\item[\code{auxData}] data frame with auxiliary data values

\item[\code{metaData}] data frame with metainformation on the entry

\item[\code{txtData}] data frame with infromation on data

\item[\code{weights}] data frame with values of weights
\end{ldescription}
\end{Arguments}
%
\begin{SeeAlso}\relax
Other add-my-pet functions: \code{\LinkA{estim\_pars}{estim.Rul.pars}};
\code{\LinkA{mydata\_pets}{mydata.Rul.pets}}; \code{\LinkA{petregr\_f}{petregr.Rul.f}};
\code{\LinkA{predict\_pets}{predict.Rul.pets}}; \code{\LinkA{printpar}{printpar}};
\code{\LinkA{printprd}{printprd}}
\end{SeeAlso}
%
\begin{Examples}
\begin{ExampleCode}
results_pets(par, metaPar, txtPar, data, auxData, metaData, txtData, weights)
\end{ExampleCode}
\end{Examples}
\inputencoding{utf8}
\HeaderA{rmpseudodata}{Removes pseudodata information from inputed data structures}{rmpseudodata}
%
\begin{Description}\relax
Removes pseudodata information from inputed data structures
\end{Description}
%
\begin{Usage}
\begin{verbatim}
rmpseudodata(data = list())
\end{verbatim}
\end{Usage}
%
\begin{Arguments}
\begin{ldescription}
\item[\code{data}] structure with "psd" field to be removed
\end{ldescription}
\end{Arguments}
%
\begin{Value}
structure with "psd" field removed
\end{Value}
%
\begin{SeeAlso}\relax
Other add-my-pet auxiliary functions: \code{\LinkA{addchem}{addchem}};
\code{\LinkA{addpseudodata}{addpseudodata}}; \code{\LinkA{fieldnm\_wtxt}{fieldnm.Rul.wtxt}};
\code{\LinkA{fieldnmnst}{fieldnmnst}}; \code{\LinkA{get\_d\_V}{get.Rul.d.Rul.V}};
\code{\LinkA{parscomp}{parscomp}}; \code{\LinkA{predict\_pseudodata}{predict.Rul.pseudodata}};
\code{\LinkA{print\_filterflag}{print.Rul.filterflag}}; \code{\LinkA{setweights}{setweights}}
\end{SeeAlso}
%
\begin{Examples}
\begin{ExampleCode}
data <- rmpseudodata(data)
\end{ExampleCode}
\end{Examples}
\inputencoding{utf8}
\HeaderA{setweights}{Sets automatically the weights for the data (to be used in a regression)}{setweights}
%
\begin{Description}\relax
computes weights for given data and adds it to the weight structure
\end{Description}
%
\begin{Usage}
\begin{verbatim}
setweights(data, weights = list())
\end{verbatim}
\end{Usage}
%
\begin{Arguments}
\begin{ldescription}
\item[\code{data}] structure with data

\item[\code{weights}] structure with weights
\end{ldescription}
\end{Arguments}
%
\begin{Details}\relax
computes weights for given data and adds it to the weight structure
for the zero-variate data y, the weight will be
\deqn{min(100, 1/ max(10^-6, y) ^2 \right)}{}
for the uni-variate data y, the weight will be
\deqn{1/ N \bar{y}^2}{}
\end{Details}
%
\begin{Value}
structure with weights
\end{Value}
%
\begin{SeeAlso}\relax
Other add-my-pet auxiliary functions: \code{\LinkA{addchem}{addchem}};
\code{\LinkA{addpseudodata}{addpseudodata}}; \code{\LinkA{fieldnm\_wtxt}{fieldnm.Rul.wtxt}};
\code{\LinkA{fieldnmnst}{fieldnmnst}}; \code{\LinkA{get\_d\_V}{get.Rul.d.Rul.V}};
\code{\LinkA{parscomp}{parscomp}}; \code{\LinkA{predict\_pseudodata}{predict.Rul.pseudodata}};
\code{\LinkA{print\_filterflag}{print.Rul.filterflag}};
\code{\LinkA{rmpseudodata}{rmpseudodata}}
\end{SeeAlso}
%
\begin{Examples}
\begin{ExampleCode}
setweights(data)
\end{ExampleCode}
\end{Examples}
\inputencoding{utf8}
\HeaderA{tempcorr}{Temperature correction}{tempcorr}
%
\begin{Description}\relax
Calculates the factor with which physiological rates should be multiplied to go from a reference temperature to a given temperature
\end{Description}
%
\begin{Usage}
\begin{verbatim}
tempcorr(Temp, T_1, T_A, T_L = NA, T_AL = NA, T_H = NA, T_AH = NA)
\end{verbatim}
\end{Usage}
%
\begin{Arguments}
\begin{ldescription}
\item[\code{Temp}] vector with temperatures (in Kelvin)

\item[\code{T\_1}] scalar with reference temperature (in Kelvin)

\item[\code{T\_A}] scalar with Arrhenius temperature (in Kelvin)

\item[\code{T\_L}] optional scalar with lower boundary of temperature range (in Kelvin)

\item[\code{T\_AL}] optional scalar with Arrhenius temperature for lower boundary of temperature range (in Kelvin)

\item[\code{T\_H}] optional scalar with upper boundary of temperature range (in Kelvin)

\item[\code{T\_AH}] optional scalar with Arrhenius temperature for upper boundary of temperature range (in Kelvin)
\end{ldescription}
\end{Arguments}
%
\begin{Details}\relax
Temperature impacts metabolic rates. This impact, in its most simplest way (1 parameter), is modeled by multiplying all the time-dependent parameters by a correction factor:
\deqn{\exp\left(\frac{T_A}{T_1} - \frac{T_A}{T}\right)}{}
For a more detailed modeling one can multiply with an extra fraction \eqn{s(T_1)/s(T)}{} with (3 parameters):
\deqn{s(T) = 1 + \exp\left(\frac{T_{AL}}{T} - \frac{T_{AL}}{T_L}\right)}{}
or (5 parameters)
\deqn{s(T) = 1 + \exp\left(\frac{T_{AL}}{T} - \frac{T_{AL}}{T_L}\right) + \exp\left(\frac{T_{AH}}{T_H} - \frac{T_{AH}}{T}\right)}{}
\end{Details}
%
\begin{Value}
vector with temperature correction factors that affect all rates
\end{Value}
%
\begin{Examples}
\begin{ExampleCode}
tempcorr(c(330, 331, 332), 320, T_A = 12000, T_L = 277, T_H = 331, T_AL = 20000, T_AH = 190000)
\end{ExampleCode}
\end{Examples}
\inputencoding{utf8}
\HeaderA{warning\_std}{Warns of unreasonable parameters for the standard DEB model without acceleration}{warning.Rul.std}
%
\begin{Description}\relax
Checks if parameter values are in the reasonable part of the parameter space
of standard DEB model without acceleration, produces warnings. Meant to be run after the estimation procedure
\end{Description}
%
\begin{Usage}
\begin{verbatim}
warning_std(par)
\end{verbatim}
\end{Usage}
%
\begin{Arguments}
\begin{ldescription}
\item[\code{par}] data frame with parameter values
\end{ldescription}
\end{Arguments}
%
\begin{SeeAlso}\relax
Other filter functions: \code{\LinkA{filter\_std}{filter.Rul.std}};
\code{\LinkA{reach\_birth}{reach.Rul.birth}}
\end{SeeAlso}
%
\begin{Examples}
\begin{ExampleCode}
warning_std(par)
\end{ExampleCode}
\end{Examples}
\printindex{}
\end{document}
