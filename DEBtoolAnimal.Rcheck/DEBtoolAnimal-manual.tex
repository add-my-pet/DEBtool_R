\nonstopmode{}
\documentclass[a4paper]{book}
\usepackage[times,inconsolata,hyper]{Rd}
\usepackage{makeidx}
\usepackage[utf8,latin1]{inputenc}
% \usepackage{graphicx} % @USE GRAPHICX@
\makeindex{}
\begin{document}
\chapter*{}
\begin{center}
{\textbf{\huge Package `DEBtoolAnimal'}}
\par\bigskip{\large \today}
\end{center}
\begin{description}
\raggedright{}
\item[Type]\AsIs{Package}
\item[Title]\AsIs{DEB functions fot an animal}
\item[Version]\AsIs{0.1}
\item[Date]\AsIs{2015-09-30}
\item[Author]\AsIs{Goncalo M. Marques }\email{goncalo.marques@tecnico.ulisboa.pt}\AsIs{}
\item[Maintainer]\AsIs{Goncalo M. Marques }\email{goncalo.marques@tecnico.ulisboa.pt}\AsIs{}
\item[Description]\AsIs{DEB based functions for the std and abj models for animals.}
\item[License]\AsIs{GPL}
\item[LazyData]\AsIs{TRUE}
\item[NeedsCompilation]\AsIs{no}
\end{description}
\Rdcontents{\R{} topics documented:}
\inputencoding{utf8}
\HeaderA{beta0}{Particular incomplete beta function}{beta0}
%
\begin{Description}\relax
particular incomplete beta function:
\end{Description}
%
\begin{Usage}
\begin{verbatim}
beta0(x0, x1)
\end{verbatim}
\end{Usage}
%
\begin{Arguments}
\begin{ldescription}
\item[\code{x0}] scalar with lower boundary for integration

\item[\code{x1}] scalar with upper boundary for integration
\end{ldescription}
\end{Arguments}
%
\begin{Value}
scalar with particular incomple beta function
\end{Value}
%
\begin{SeeAlso}\relax
Other miscelaneous functions: \code{\LinkA{C2K}{C2K}};
\code{\LinkA{K2C}{K2C}}; \code{\LinkA{get\_lb}{get.Rul.lb}};
\code{\LinkA{get\_ue0}{get.Rul.ue0}}; \code{\LinkA{tempcorr}{tempcorr}}
\end{SeeAlso}
%
\begin{Examples}
\begin{ExampleCode}
beta0(0.1, 0.2)
\end{ExampleCode}
\end{Examples}
\inputencoding{utf8}
\HeaderA{C2K}{Conversion of Celsius to Kelvin}{C2K}
%
\begin{Description}\relax
Computes Kelvin from temperatures defined in Celsius
\end{Description}
%
\begin{Usage}
\begin{verbatim}
C2K(C)
\end{verbatim}
\end{Usage}
%
\begin{Arguments}
\begin{ldescription}
\item[\code{C}] numeric temperature in degrees Celsius
\end{ldescription}
\end{Arguments}
%
\begin{Value}
temperature in Kelvin
\end{Value}
%
\begin{SeeAlso}\relax
Other miscelaneous functions: \code{\LinkA{K2C}{K2C}};
\code{\LinkA{beta0}{beta0}}; \code{\LinkA{get\_lb}{get.Rul.lb}};
\code{\LinkA{get\_ue0}{get.Rul.ue0}}; \code{\LinkA{tempcorr}{tempcorr}}
\end{SeeAlso}
%
\begin{Examples}
\begin{ExampleCode}
C2K(20)
\end{ExampleCode}
\end{Examples}
\inputencoding{utf8}
\HeaderA{dget\_lambdab2}{Computes derivative d delta/dx}{dget.Rul.lambdab2}
%
\begin{Description}\relax
Obtains the derivative d delta/dx from lambdab, xb and k.
\end{Description}
%
\begin{Usage}
\begin{verbatim}
dget_lambdab2(x, delta, pars)
\end{verbatim}
\end{Usage}
%
\begin{Arguments}
\begin{ldescription}
\item[\code{x}] scalar x = g/(g + e)

\item[\code{delta}] scalar delta = x e\_H/ (1 - kap)g

\item[\code{pars}] data.frame with lambdab, xb, xb3 (xb\textasciicircum{}1/3), k
\end{ldescription}
\end{Arguments}
%
\begin{Value}
scalar with derivative value d delta/ dx
\end{Value}
%
\begin{SeeAlso}\relax
Other scaled get functions: \code{\LinkA{fnget\_lambdab2}{fnget.Rul.lambdab2}};
\code{\LinkA{get\_lambdab2}{get.Rul.lambdab2}}; \code{\LinkA{get\_lambdab}{get.Rul.lambdab}}
\end{SeeAlso}
%
\begin{Examples}
\begin{ExampleCode}
dget_lambdab2(10^(-6), 0, c(lambdab = 0.003, xb = 10/11, xb3 = (10/11)^(1/3), k = 1))
\end{ExampleCode}
\end{Examples}
\inputencoding{utf8}
\HeaderA{fnget\_lambdab2}{Computes initial scaled reserve}{fnget.Rul.lambdab2}
%
\begin{Description}\relax
Obtains scaled length at birth, given the scaled reserve density at birth.
\end{Description}
%
\begin{Usage}
\begin{verbatim}
fnget_lambdab2(lambdab, pars)
\end{verbatim}
\end{Usage}
%
\begin{Arguments}
\begin{ldescription}
\item[\code{p}] 3-vector with parameters: g, k, vv\_H\textasciicircum{}b (see below)

\item[\code{eb}] optional scalar with scaled reserve density at birth (default eb = 1)

\item[\code{lambdab0}] optional scalar with initial estimate for scaled length at birth (default lambdab0: lambdab for k = 1)
\end{ldescription}
\end{Arguments}
%
\begin{Value}
scalar with scaled length at birth (lambdab) and indicator equals 1 if successful, 0 otherwise (info)
\end{Value}
%
\begin{SeeAlso}\relax
Other scaled get functions: \code{\LinkA{dget\_lambdab2}{dget.Rul.lambdab2}};
\code{\LinkA{get\_lambdab2}{get.Rul.lambdab2}}; \code{\LinkA{get\_lambdab}{get.Rul.lambdab}}
\end{SeeAlso}
%
\begin{Examples}
\begin{ExampleCode}
get_lambdab(c(10, 1, 0.01), 1, 0.1)
\end{ExampleCode}
\end{Examples}
\inputencoding{utf8}
\HeaderA{get\_lambdab}{Computes initial scaled reserve}{get.Rul.lambdab}
%
\begin{Description}\relax
Obtains scaled length at birth, given the scaled reserve density at birth.
\end{Description}
%
\begin{Usage}
\begin{verbatim}
get_lambdab(p, eb, lambdab0 = NA)
\end{verbatim}
\end{Usage}
%
\begin{Arguments}
\begin{ldescription}
\item[\code{p}] 3-vector with parameters: g, k, vv\_H\textasciicircum{}b (see below)

\item[\code{eb}] optional scalar with scaled reserve density at birth (default eb = 1)

\item[\code{lambdab0}] optional scalar with initial estimate for scaled length at birth (default lambdab0: lambdab for k = 1)
\end{ldescription}
\end{Arguments}
%
\begin{Value}
scalar with scaled length at birth (lambdab) and indicator equals 1 if successful, 0 otherwise (info)
\end{Value}
%
\begin{SeeAlso}\relax
Other scaled get functions: \code{\LinkA{dget\_lambdab2}{dget.Rul.lambdab2}};
\code{\LinkA{fnget\_lambdab2}{fnget.Rul.lambdab2}}; \code{\LinkA{get\_lambdab2}{get.Rul.lambdab2}}
\end{SeeAlso}
%
\begin{Examples}
\begin{ExampleCode}
get_lambdab(c(10, 1, 0.01), 1, 0.1)
\end{ExampleCode}
\end{Examples}
\inputencoding{utf8}
\HeaderA{get\_lambdab2}{Computes initial scaled reserve}{get.Rul.lambdab2}
%
\begin{Description}\relax
Obtains scaled length at birth, given the scaled reserve density at birth.
\end{Description}
%
\begin{Usage}
\begin{verbatim}
get_lambdab2(p, eb, lambdab0 = NA)
\end{verbatim}
\end{Usage}
%
\begin{Arguments}
\begin{ldescription}
\item[\code{p}] 3-vector with parameters: g, k, vv\_H\textasciicircum{}b (see below)

\item[\code{eb}] optional scalar with scaled reserve density at birth (default eb = 1)

\item[\code{lambdab0}] optional scalar with initial estimate for scaled length at birth (default lambdab0: lambdab for k = 1)
\end{ldescription}
\end{Arguments}
%
\begin{Value}
scalar with scaled length at birth (lambdab) and indicator equals 1 if successful, 0 otherwise (info)
\end{Value}
%
\begin{SeeAlso}\relax
Other scaled get functions: \code{\LinkA{dget\_lambdab2}{dget.Rul.lambdab2}};
\code{\LinkA{fnget\_lambdab2}{fnget.Rul.lambdab2}}; \code{\LinkA{get\_lambdab}{get.Rul.lambdab}}
\end{SeeAlso}
%
\begin{Examples}
\begin{ExampleCode}
get_lambdab(c(10, 1, 0.01), 1, 0.1)
\end{ExampleCode}
\end{Examples}
\inputencoding{utf8}
\HeaderA{get\_lb}{Computes initial scaled reserve}{get.Rul.lb}
%
\begin{Description}\relax
particular incomplete beta function:
\end{Description}
%
\begin{Usage}
\begin{verbatim}
get_lb(p, eb, lb0 = NA)
\end{verbatim}
\end{Usage}
%
\begin{Arguments}
\begin{ldescription}
\item[\code{x0}] scalar with lower boundary for integration

\item[\code{x1}] scalar with upper boundary for integration
\end{ldescription}
\end{Arguments}
%
\begin{Value}
scalar with particular incomple beta function
\end{Value}
%
\begin{SeeAlso}\relax
Other miscelaneous functions: \code{\LinkA{C2K}{C2K}};
\code{\LinkA{K2C}{K2C}}; \code{\LinkA{beta0}{beta0}};
\code{\LinkA{get\_ue0}{get.Rul.ue0}}; \code{\LinkA{tempcorr}{tempcorr}}
\end{SeeAlso}
%
\begin{Examples}
\begin{ExampleCode}
beta0(0.1, 0.2)
\end{ExampleCode}
\end{Examples}
\inputencoding{utf8}
\HeaderA{get\_ue0}{Computes initial scaled reserve}{get.Rul.ue0}
%
\begin{Description}\relax
particular incomplete beta function:
\end{Description}
%
\begin{Usage}
\begin{verbatim}
get_ue0(p, eb, lb0)
\end{verbatim}
\end{Usage}
%
\begin{Arguments}
\begin{ldescription}
\item[\code{x0}] scalar with lower boundary for integration

\item[\code{x1}] scalar with upper boundary for integration
\end{ldescription}
\end{Arguments}
%
\begin{Value}
scalar with particular incomple beta function
\end{Value}
%
\begin{SeeAlso}\relax
Other miscelaneous functions: \code{\LinkA{C2K}{C2K}};
\code{\LinkA{K2C}{K2C}}; \code{\LinkA{beta0}{beta0}};
\code{\LinkA{get\_lb}{get.Rul.lb}}; \code{\LinkA{tempcorr}{tempcorr}}
\end{SeeAlso}
%
\begin{Examples}
\begin{ExampleCode}
beta0(0.1, 0.2)
\end{ExampleCode}
\end{Examples}
\inputencoding{utf8}
\HeaderA{K2C}{Conversion of Kelvin to Celsius}{K2C}
%
\begin{Description}\relax
Computes Celsius from temperatures given in Kelvin
\end{Description}
%
\begin{Usage}
\begin{verbatim}
K2C(K)
\end{verbatim}
\end{Usage}
%
\begin{Arguments}
\begin{ldescription}
\item[\code{K}] numeric temperature in degrees Kelvin
\end{ldescription}
\end{Arguments}
%
\begin{Value}
temperature in Kelvin
\end{Value}
%
\begin{SeeAlso}\relax
Other miscelaneous functions: \code{\LinkA{C2K}{C2K}};
\code{\LinkA{beta0}{beta0}}; \code{\LinkA{get\_lb}{get.Rul.lb}};
\code{\LinkA{get\_ue0}{get.Rul.ue0}}; \code{\LinkA{tempcorr}{tempcorr}}
\end{SeeAlso}
%
\begin{Examples}
\begin{ExampleCode}
K2C(293.15)
\end{ExampleCode}
\end{Examples}
\inputencoding{utf8}
\HeaderA{tempcorr}{Conversion of Kelvin to Celsius}{tempcorr}
%
\begin{Description}\relax
Calculates the factor with which physiological rates should be multiplied to go from a reference temperature to a given temperature
\end{Description}
%
\begin{Usage}
\begin{verbatim}
tempcorr(Temp, T_1, Tpars)
\end{verbatim}
\end{Usage}
%
\begin{Arguments}
\begin{ldescription}
\item[\code{T\_1}] scalar with reference temperature

\item[\code{Tpars}] 1-, 3- or 5-vector with temperature parameters

\item[\code{T}] vector with new temperatures
\end{ldescription}
\end{Arguments}
%
\begin{Value}
vector with temperature correction factors that affect all rates
\end{Value}
%
\begin{SeeAlso}\relax
Other miscelaneous functions: \code{\LinkA{C2K}{C2K}};
\code{\LinkA{K2C}{K2C}}; \code{\LinkA{beta0}{beta0}};
\code{\LinkA{get\_lb}{get.Rul.lb}}; \code{\LinkA{get\_ue0}{get.Rul.ue0}}
\end{SeeAlso}
%
\begin{Examples}
\begin{ExampleCode}
tempcorr(c(330, 331, 332), 320, c(12000, 277, 318, 20000, 190000))
\end{ExampleCode}
\end{Examples}
\printindex{}
\end{document}
