\nonstopmode{}
\documentclass[a4paper]{book}
\usepackage[times,inconsolata,hyper]{Rd}
\usepackage{makeidx}
\usepackage[utf8,latin1]{inputenc}
% \usepackage{graphicx} % @USE GRAPHICX@
\makeindex{}
\begin{document}
\chapter*{}
\begin{center}
{\textbf{\huge Package `DEBtoolAnimal'}}
\par\bigskip{\large \today}
\end{center}
\begin{description}
\raggedright{}
\item[Type]\AsIs{Package}
\item[Title]\AsIs{DEB functions for an animal}
\item[Version]\AsIs{0.1}
\item[Date]\AsIs{2015-09-30}
\item[Author]\AsIs{Goncalo M. Marques }\email{goncalo.marques@tecnico.ulisboa.pt}\AsIs{}
\item[Maintainer]\AsIs{Goncalo M. Marques }\email{goncalo.marques@tecnico.ulisboa.pt}\AsIs{}
\item[Description]\AsIs{DEB based functions for the  std (standard) and abj (with acceleration) models for animals.}
\item[License]\AsIs{GPL}
\item[LazyData]\AsIs{TRUE}
\item[NeedsCompilation]\AsIs{no}
\end{description}
\Rdcontents{\R{} topics documented:}
\inputencoding{utf8}
\HeaderA{addpseudodata}{Adds pseudodata information into inputed data structures}{addpseudodata}
%
\begin{Description}\relax
Adds the pseudodata information and weights for purposes of the regression
\end{Description}
%
\begin{Usage}
\begin{verbatim}
addpseudodata(data = list(), units = list(), label = list(),
  weights = list())
\end{verbatim}
\end{Usage}
%
\begin{Arguments}
\begin{ldescription}
\item[\code{data}] structure with data

\item[\code{units}] structure with data units

\item[\code{label}] structure with data labels

\item[\code{weights}] structure with weights
\end{ldescription}
\end{Arguments}
%
\begin{Value}
structures with data, units, labael and weights
\end{Value}
%
\begin{SeeAlso}\relax
Other add-my-pet auxiliary functions: \code{\LinkA{setweights}{setweights}}
\end{SeeAlso}
%
\begin{Examples}
\begin{ExampleCode}
list[data, units, label, weight] <- addpseudodata();
\end{ExampleCode}
\end{Examples}
\inputencoding{utf8}
\HeaderA{beta0}{Particular incomplete beta function}{beta0}
%
\begin{Description}\relax
particular incomplete beta function:
\end{Description}
%
\begin{Usage}
\begin{verbatim}
beta0(x0, x1)
\end{verbatim}
\end{Usage}
%
\begin{Arguments}
\begin{ldescription}
\item[\code{x0}] scalar with lower boundary for integration

\item[\code{x1}] scalar with upper boundary for integration
\end{ldescription}
\end{Arguments}
%
\begin{Details}\relax
Computes
\deqn{B_{x_1}\left(\frac{4}{3},0\right) - B_{x_0}\left(\frac{4}{3},0\right) = \int_{x_0}^{x_1} t^{4/3-1} (1-t)^{-1} dt}{}
To be used in the computation of the age at birth (or related quantities) for an egg.
\end{Details}
%
\begin{Value}
scalar with particular incomple beta function
\end{Value}
%
\begin{SeeAlso}\relax
Other miscellaneous functions: \code{\LinkA{C2K}{C2K}};
\code{\LinkA{K2C}{K2C}}
\end{SeeAlso}
%
\begin{Examples}
\begin{ExampleCode}
beta0(0.1, 0.2)
\end{ExampleCode}
\end{Examples}
\inputencoding{utf8}
\HeaderA{C2K}{Conversion of Celsius to Kelvin}{C2K}
%
\begin{Description}\relax
Converts temperature in degrees Celsius to Kelvin
\end{Description}
%
\begin{Usage}
\begin{verbatim}
C2K(C)
\end{verbatim}
\end{Usage}
%
\begin{Arguments}
\begin{ldescription}
\item[\code{C}] numeric temperature in degrees Celsius
\end{ldescription}
\end{Arguments}
%
\begin{Value}
temperature in Kelvin
\end{Value}
%
\begin{SeeAlso}\relax
Other miscellaneous functions: \code{\LinkA{K2C}{K2C}};
\code{\LinkA{beta0}{beta0}}
\end{SeeAlso}
%
\begin{Examples}
\begin{ExampleCode}
C2K(20)
\end{ExampleCode}
\end{Examples}
\inputencoding{utf8}
\HeaderA{dget\_lbarb2}{Computes derivative d delta/dx}{dget.Rul.lbarb2}
%
\begin{Description}\relax
Obtains the derivative d delta/dx from lbarb, xb and k.
\end{Description}
%
\begin{Usage}
\begin{verbatim}
dget_lbarb2(x, delta, pars)
\end{verbatim}
\end{Usage}
%
\begin{Arguments}
\begin{ldescription}
\item[\code{x}] scalar x = g/(g + e)

\item[\code{delta}] scalar delta = x e\_H/ (1 - kap)g

\item[\code{pars}] data.frame with lbarb, xb, xb3 (xb\textasciicircum{}1/3), k
\end{ldescription}
\end{Arguments}
%
\begin{Value}
scalar with derivative value d delta/ dx
\end{Value}
%
\begin{SeeAlso}\relax
Other scaled get functions: \code{\LinkA{fnget\_lbarb2}{fnget.Rul.lbarb2}};
\code{\LinkA{get\_lbarb2}{get.Rul.lbarb2}}; \code{\LinkA{get\_lbarb}{get.Rul.lbarb}};
\code{\LinkA{get\_lb}{get.Rul.lb}};
\code{\LinkA{initial\_scaled\_reserve}{initial.Rul.scaled.Rul.reserve}}
\end{SeeAlso}
%
\begin{Examples}
\begin{ExampleCode}
dget_lbarb2(10^(-6), 0, c(lbarb = 0.003, xb = 10/11, xb3 = (10/11)^(1/3), k = 1))
\end{ExampleCode}
\end{Examples}
\inputencoding{utf8}
\HeaderA{estim\_options}{Sets options for estim\_pars}{estim.Rul.options}
%
\begin{Description}\relax
Sets options for estimation one by one
\end{Description}
%
\begin{Usage}
\begin{verbatim}
estim_options(key = "inexistent", val = "")
\end{verbatim}
\end{Usage}
%
\begin{Arguments}
\begin{ldescription}
\item[\code{key}] string with option to set

\item[\code{val}] value of the option
\end{ldescription}
\end{Arguments}
%
\begin{Details}\relax
no input: print values to screen

one input:

* "default": sets options at default values
* any other key (see below): print value to screen

two inputs:

* "filter": 1 - use filter (default); 0 - do not;
* "pars\_init\_method":
0 - get initial estimates from automatized computation (default)
1 - read initial estimates from .mat file (for continuation)
2 - read initial estimates from pars\_init file
* "pseudodata\_pets":
0 - put pseudodata together with data (default)
1 - put it apart (only for multispecies estimation)
* "results\_output":
0 - prints results to screen (default)
1 - prints results to screen, saves to .mat file
2 - saves data to .mat file and graphs to .png files
(prints results to screen using a customized results file when it exists)
* "method": "nm" - use Nelder-Mead method; "no" - do not estimate;

for other options see corresponding options file of the method (e.g. nmregr\_options)
\end{Details}
%
\begin{SeeAlso}\relax
Other regression functions: \code{\LinkA{nmregr\_options}{nmregr.Rul.options}}
\end{SeeAlso}
%
\begin{Examples}
\begin{ExampleCode}
estim_options("default")
\end{ExampleCode}
\end{Examples}
\inputencoding{utf8}
\HeaderA{estim\_pars}{Estimates parameters}{estim.Rul.pars}
%
\begin{Description}\relax
Runs the entire estimation procedures: gets the parameters, gets the data, intiates the estimation procedure and sends the results for handling
\end{Description}
%
\begin{Usage}
\begin{verbatim}
estim_pars()
\end{verbatim}
\end{Usage}
%
\begin{SeeAlso}\relax
Other add-my-pet functions: \code{\LinkA{mydata\_pets}{mydata.Rul.pets}}
\end{SeeAlso}
%
\begin{Examples}
\begin{ExampleCode}
estim_pars()
\end{ExampleCode}
\end{Examples}
\inputencoding{utf8}
\HeaderA{fnget\_lbarb2}{Computes f using the ode solver for delta(x), for finding lbarb}{fnget.Rul.lbarb2}
%
\begin{Description}\relax
Computes f using the ode solver for delta(x), for finding lbarb.
\end{Description}
%
\begin{Usage}
\begin{verbatim}
fnget_lbarb2(lbarb, pars)
\end{verbatim}
\end{Usage}
%
\begin{Arguments}
\begin{ldescription}
\item[\code{lbarb}] scalar with scaled length at birth (lbarb = lb/ g)

\item[\code{pars}] data.frame with lbarb, xb, xb3 (xb\textasciicircum{}1/3), k
\end{ldescription}
\end{Arguments}
%
\begin{Value}
scalar with function f which when zero indicates lbarb
\end{Value}
%
\begin{SeeAlso}\relax
Other scaled get functions: \code{\LinkA{dget\_lbarb2}{dget.Rul.lbarb2}};
\code{\LinkA{get\_lbarb2}{get.Rul.lbarb2}}; \code{\LinkA{get\_lbarb}{get.Rul.lbarb}};
\code{\LinkA{get\_lb}{get.Rul.lb}};
\code{\LinkA{initial\_scaled\_reserve}{initial.Rul.scaled.Rul.reserve}}
\end{SeeAlso}
%
\begin{Examples}
\begin{ExampleCode}
fnget_lbarb2(0.03, c(xb = 10/11, xb3 = (10/11)^(1/3), vbarHb = 0.001, k = 1))
\end{ExampleCode}
\end{Examples}
\inputencoding{utf8}
\HeaderA{get\_lb}{Computes scaled length at birth}{get.Rul.lb}
%
\begin{Description}\relax
Obtains scaled length at birth, given the scaled reserve density at birth.
\end{Description}
%
\begin{Usage}
\begin{verbatim}
get_lb(pars, eb = 1, lb0 = as.numeric(pars[3]^(1/3)))
\end{verbatim}
\end{Usage}
%
\begin{Arguments}
\begin{ldescription}
\item[\code{pars}] 3-vector with parameters: g, k, v\_H\textasciicircum{}b

\item[\code{eb}] optional scalar with scaled reserve density at birth (default eb = 1)

\item[\code{lb0}] optional scalar with initial estimate for scaled length at birth (default lb0: lb for k = 1)
\end{ldescription}
\end{Arguments}
%
\begin{Details}\relax
The theory behind get\_lb, get\_tb and get\_ue0 is discussed in http://www.bio.vu.nl/thb/research/bib/Kooy2009b.html.
Solves \eqn{y(x_b) = y_b}{}  for \eqn{l_b}{} with explicit solution for \eqn{y(x)}{}
\deqn{y(x) = \frac{x e_H}{1-kap} = x g \frac{u_H}{l^3}}{}
and \eqn{y_b = x_b g u_H^b/ ((1-kap)l_b^3)}{}
\deqn{\frac{d}{dx} y = r(x) - y s(x)}{}
with solution \eqn{y(x) = v(x) \int_0^x r(x')/ v(x') dx'}{}
and \eqn{v(x) = exp(- \int_0^x s(x') dx')}{}.
A Newton Raphson scheme is used with Euler integration, starting from an optional initial value.
Shooting method: get\_lb2.
In case of no convergence, get\_lb2 is run automatically as backup.
Consider the application of get\_lb\_foetus for an alternative initial value.
\end{Details}
%
\begin{Value}
scalar with scaled length at birth (lb) and indicator equals 1 if successful, 0 otherwise (info)
\end{Value}
%
\begin{SeeAlso}\relax
Other scaled get functions: \code{\LinkA{dget\_lbarb2}{dget.Rul.lbarb2}};
\code{\LinkA{fnget\_lbarb2}{fnget.Rul.lbarb2}}; \code{\LinkA{get\_lbarb2}{get.Rul.lbarb2}};
\code{\LinkA{get\_lbarb}{get.Rul.lbarb}};
\code{\LinkA{initial\_scaled\_reserve}{initial.Rul.scaled.Rul.reserve}}
\end{SeeAlso}
%
\begin{Examples}
\begin{ExampleCode}
get_lb(c(g = 10, k = 1, vHb = 0.5), 1)
\end{ExampleCode}
\end{Examples}
\inputencoding{utf8}
\HeaderA{get\_lbarb}{Computes scaled length at birth lbarb}{get.Rul.lbarb}
%
\begin{Description}\relax
Obtains scaled length at birth, given the scaled reserve density at birth.
\end{Description}
%
\begin{Usage}
\begin{verbatim}
get_lbarb(pars, eb = 1, lbarb0 = NA)
\end{verbatim}
\end{Usage}
%
\begin{Arguments}
\begin{ldescription}
\item[\code{pars}] 3-vector with parameters: g, k, vbar\_H\textasciicircum{}b

\item[\code{eb}] optional scalar with scaled reserve density at birth (default eb = 1)

\item[\code{lbarb0}] optional scalar with initial estimate for scaled length at birth (default lbarb0: lbarb for k = 1)
\end{ldescription}
\end{Arguments}
%
\begin{Value}
scalar with scaled length at birth (lbarb) and indicator equals 1 if successful, 0 otherwise (info)
\end{Value}
%
\begin{SeeAlso}\relax
Other scaled get functions: \code{\LinkA{dget\_lbarb2}{dget.Rul.lbarb2}};
\code{\LinkA{fnget\_lbarb2}{fnget.Rul.lbarb2}}; \code{\LinkA{get\_lbarb2}{get.Rul.lbarb2}};
\code{\LinkA{get\_lb}{get.Rul.lb}};
\code{\LinkA{initial\_scaled\_reserve}{initial.Rul.scaled.Rul.reserve}}
\end{SeeAlso}
%
\begin{Examples}
\begin{ExampleCode}
get_lbarb(c(g = 10, k = 1, vbarHb = 0.0005), 1)
\end{ExampleCode}
\end{Examples}
\inputencoding{utf8}
\HeaderA{get\_lbarb2}{Computes initial scaled reserve}{get.Rul.lbarb2}
%
\begin{Description}\relax
Obtains scaled length at birth, given the scaled reserve density at birth. Like get\_lbarb, but uses a shooting method in 1 variable.
\end{Description}
%
\begin{Usage}
\begin{verbatim}
get_lbarb2(pars, eb = NA)
\end{verbatim}
\end{Usage}
%
\begin{Arguments}
\begin{ldescription}
\item[\code{pars}] 3-vector with parameters: g, k, vbar\_H\textasciicircum{}b

\item[\code{eb}] optional scalar with scaled reserve density at birth (default eb = 1)
\end{ldescription}
\end{Arguments}
%
\begin{Value}
scalar with scaled length at birth (lbarb) and indicator equals 1 if successful, 0 otherwise (info)
\end{Value}
%
\begin{SeeAlso}\relax
Other scaled get functions: \code{\LinkA{dget\_lbarb2}{dget.Rul.lbarb2}};
\code{\LinkA{fnget\_lbarb2}{fnget.Rul.lbarb2}}; \code{\LinkA{get\_lbarb}{get.Rul.lbarb}};
\code{\LinkA{get\_lb}{get.Rul.lb}};
\code{\LinkA{initial\_scaled\_reserve}{initial.Rul.scaled.Rul.reserve}}
\end{SeeAlso}
%
\begin{Examples}
\begin{ExampleCode}
get_lbarb2(c(g = 10, k = 1, vbarHb = 0.01), 1)
\end{ExampleCode}
\end{Examples}
\inputencoding{utf8}
\HeaderA{get\_ubarE0}{Computes initial scaled reserve density at birth}{get.Rul.ubarE0}
%
\begin{Description}\relax
Obtains the initial scaled reserve given the scaled reserve density at birth.
Function get\_ue0 does so for eggs, get\_ue0\_foetus for foetuses.
Specification of length at birth as third input by-passes its computation,
so if you want to specify an initial value for this quantity, you should use get\_lb directly.
\end{Description}
%
\begin{Usage}
\begin{verbatim}
get_ubarE0(g = NA, k = NA, vbarHb = NA, eb = 1, lbarb = NA)
\end{verbatim}
\end{Usage}
%
\begin{Arguments}
\begin{ldescription}
\item[\code{g}] energy investment ratio

\item[\code{k}] maintenance ratio

\item[\code{vbarHb}] rescaled maturity volume at birth

\item[\code{eb}] optional scalar with scaled reserbe density at birth

\item[\code{lbarb}] optional scalar with scaled length at birth
\end{ldescription}
\end{Arguments}
%
\begin{Value}
scalar with particular incomple beta function
\end{Value}
%
\begin{SeeAlso}\relax
Other get functions: \code{\LinkA{get\_ue0}{get.Rul.ue0}}
\end{SeeAlso}
%
\begin{Examples}
\begin{ExampleCode}
get_ubarE0(g = 10, lbarb = 0.01)
get_ubarE0(g = 10, k = 0.7, vbarHb = 5e-4)
\end{ExampleCode}
\end{Examples}
\inputencoding{utf8}
\HeaderA{get\_ue0}{Computes initial scaled reserve}{get.Rul.ue0}
%
\begin{Description}\relax
Obtains the initial scaled reserve given the scaled reserve density at birth.
Function get\_ue0 does so for eggs, get\_ue0\_foetus for foetuses.
Specification of length at birth as third input by-passes its computation,
so if you want to specify an initial value for this quantity, you should use get\_lb directly.
\end{Description}
%
\begin{Usage}
\begin{verbatim}
get_ue0(pars, eb = 1, lb0 = NA)
\end{verbatim}
\end{Usage}
%
\begin{Arguments}
\begin{ldescription}
\item[\code{pars}] 1 or 3 -vector with parameters g, k\_J/ k\_M, v\_H\textasciicircum{}b, see get\_lb

\item[\code{eb}] optional scalar with scaled reserbe density at birth (default: eb = 1)

\item[\code{lb0}] optional scalar with scaled length at birth (default: lb is optained from get\_lb)
\end{ldescription}
\end{Arguments}
%
\begin{Value}
uE0 scalar with scaled reserve at t=0: \$U\_E\textasciicircum{}0 g\textasciicircum{}2 k\_M\textasciicircum{}3/ v\textasciicircum{}2\$ with \$U\_E\textasciicircum{}0 = M\_E\textasciicircum{}0/ \{J\_EAm\}\$, lb scalar with scaled length at birth and info indicator equals 1 if successful, 0 otherwise
\end{Value}
%
\begin{SeeAlso}\relax
Other get functions: \code{\LinkA{get\_ubarE0}{get.Rul.ubarE0}}
\end{SeeAlso}
%
\begin{Examples}
\begin{ExampleCode}
get_ue0(pars = c(0.42, 1, 0.066), eb = 1, lb0 = 0.4042)
\end{ExampleCode}
\end{Examples}
\inputencoding{utf8}
\HeaderA{initial\_scaled\_reserve}{Gets initial scaled reserve}{initial.Rul.scaled.Rul.reserve}
%
\begin{Description}\relax
Gets initial scaled reserve.
\end{Description}
%
\begin{Usage}
\begin{verbatim}
initial_scaled_reserve(f, pars, Lb0 = NA)
\end{verbatim}
\end{Usage}
%
\begin{Arguments}
\begin{ldescription}
\item[\code{f}] n-vector with scaled functional responses

\item[\code{pars}] 5-vector with parameters: VHb, g, kJ, kM, v

\item[\code{Lb0}] optional n-vector with lengths at birth
\end{ldescription}
\end{Arguments}
%
\begin{Value}
n-vector with initial scaled reserve: M\_E\textasciicircum{}0/ J\_EAm (U0), n-vector with length at birth (Lb) and n-vector with 1's if successful, 0's otherwise (info)
\end{Value}
%
\begin{SeeAlso}\relax
Other scaled get functions: \code{\LinkA{dget\_lbarb2}{dget.Rul.lbarb2}};
\code{\LinkA{fnget\_lbarb2}{fnget.Rul.lbarb2}}; \code{\LinkA{get\_lbarb2}{get.Rul.lbarb2}};
\code{\LinkA{get\_lbarb}{get.Rul.lbarb}}; \code{\LinkA{get\_lb}{get.Rul.lb}}
\end{SeeAlso}
%
\begin{Examples}
\begin{ExampleCode}
initial_scaled_reserve(f = c(1, 0.9), pars = c(VHb = .8, g = .42, kJ = 1.7, kM = 1.7, v = 3.24))
\end{ExampleCode}
\end{Examples}
\inputencoding{utf8}
\HeaderA{K2C}{Conversion of Kelvin to Celsius}{K2C}
%
\begin{Description}\relax
Converts temperature in Kelvin to degrees Celsius
\end{Description}
%
\begin{Usage}
\begin{verbatim}
K2C(K)
\end{verbatim}
\end{Usage}
%
\begin{Arguments}
\begin{ldescription}
\item[\code{K}] numeric temperature in degrees Kelvin
\end{ldescription}
\end{Arguments}
%
\begin{Value}
temperature in Kelvin
\end{Value}
%
\begin{SeeAlso}\relax
Other miscellaneous functions: \code{\LinkA{C2K}{C2K}};
\code{\LinkA{beta0}{beta0}}
\end{SeeAlso}
%
\begin{Examples}
\begin{ExampleCode}
K2C(293.15)
\end{ExampleCode}
\end{Examples}
\inputencoding{utf8}
\HeaderA{mydata\_my\_pet}{Sets referenced data}{mydata.Rul.my.Rul.pet}
%
\begin{Description}\relax
Sets data, pseudodata, metadata, auxdata, explanatory text, weights coefficients. Meant to be a template in add-my-pet
\end{Description}
%
\begin{Usage}
\begin{verbatim}
mydata_my_pet()
\end{verbatim}
\end{Usage}
%
\begin{Value}
list with data, auxData, metaData, txtData and weights
\end{Value}
%
\begin{Examples}
\begin{ExampleCode}
mydata_my_pet()
\end{ExampleCode}
\end{Examples}
\inputencoding{utf8}
\HeaderA{mydata\_pets}{Concatenates mydata files for several species}{mydata.Rul.pets}
%
\begin{Description}\relax
Concatenates mydata files for several species
\end{Description}
%
\begin{Usage}
\begin{verbatim}
mydata_pets()
\end{verbatim}
\end{Usage}
%
\begin{Value}
structure with data, auxData, metaData, txtData and weights for several pets
\end{Value}
%
\begin{SeeAlso}\relax
Other add-my-pet functions: \code{\LinkA{estim\_pars}{estim.Rul.pars}}
\end{SeeAlso}
%
\begin{Examples}
\begin{ExampleCode}
mydata_pets()
\end{ExampleCode}
\end{Examples}
\inputencoding{utf8}
\HeaderA{nmregr\_options}{Sets options for function nmregr}{nmregr.Rul.options}
%
\begin{Description}\relax
Sets options for estimation one by one
\end{Description}
%
\begin{Usage}
\begin{verbatim}
nmregr_options(key = "inexistent", val = "")
\end{verbatim}
\end{Usage}
%
\begin{Arguments}
\begin{ldescription}
\item[\code{key}] string with option to set

\item[\code{val}] value of the option
\end{ldescription}
\end{Arguments}
%
\begin{Details}\relax
no input: print values to screen

one input:

* "default": sets options at default values
* any other key (see below): print value to screen

two inputs:

* "report": 1 - to report steps to screen; 0 - not to;
* "max\_step\_number": maximum number of steps
* "max\_fun\_evals": maximum number of function evaluations
* "tol\_simplex": tolerance for how close the simplex points must be together to call them the same
* "tol\_tun": tolerance for how close the loss-function values must be together to call them the same
* "simplex\_size": fraction added (subtracted if negative) to the free parameters when building the simplex
\end{Details}
%
\begin{Value}
1 if input is valid key, 0 if input is unknown key
\end{Value}
%
\begin{SeeAlso}\relax
Other regression functions: \code{\LinkA{estim\_options}{estim.Rul.options}}
\end{SeeAlso}
%
\begin{Examples}
\begin{ExampleCode}
nmregr_options("default")
\end{ExampleCode}
\end{Examples}
\inputencoding{utf8}
\HeaderA{setweights}{Sets automatically the weights for the data (to be used in a regression)}{setweights}
%
\begin{Description}\relax
computes weights for given data and adds it to the weight structure
\end{Description}
%
\begin{Usage}
\begin{verbatim}
setweights(data, weights = list())
\end{verbatim}
\end{Usage}
%
\begin{Arguments}
\begin{ldescription}
\item[\code{data}] structure with data

\item[\code{weights}] structure with weights
\end{ldescription}
\end{Arguments}
%
\begin{Details}\relax
computes weights for given data and adds it to the weight structure
for the zero-variate data y the weight will be
\deqn{min(100, 1/ max(10^-6, y) ^2 \right)}{}
for the uni-variate data y the weight will be
\deqn{1/ N \bar{y}^2}{}
\end{Details}
%
\begin{Value}
structure with weights
\end{Value}
%
\begin{SeeAlso}\relax
Other add-my-pet auxiliary functions: \code{\LinkA{addpseudodata}{addpseudodata}}
\end{SeeAlso}
%
\begin{Examples}
\begin{ExampleCode}
setweights(data)
\end{ExampleCode}
\end{Examples}
\inputencoding{utf8}
\HeaderA{tempcorr}{Temperature correction}{tempcorr}
%
\begin{Description}\relax
Calculates the factor with which physiological rates should be multiplied to go from a reference temperature to a given temperature
\end{Description}
%
\begin{Usage}
\begin{verbatim}
tempcorr(Temp, T_1, T_A, T_L = NA, T_AL = NA, T_H = NA, T_AH = NA)
\end{verbatim}
\end{Usage}
%
\begin{Arguments}
\begin{ldescription}
\item[\code{Temp}] vector with temperatures (in Kelvin)

\item[\code{T\_1}] scalar with reference temperature (in Kelvin)

\item[\code{T\_A}] scalar with Arrhenius temperature (in Kelvin)

\item[\code{T\_L}] optional scalar with lower boundary of temperature range (in Kelvin)

\item[\code{T\_AL}] optional scalar with Arrhenius temperature for lower boundary of temperature range (in Kelvin)

\item[\code{T\_H}] optional scalar with upper boundary of temperature range (in Kelvin)

\item[\code{T\_AH}] optional scalar with Arrhenius temperature for upper boundary of temperature range (in Kelvin)
\end{ldescription}
\end{Arguments}
%
\begin{Details}\relax
Temperature impacts metabolic rates. This impact, in its most simplest way (1 parameter), is modeled by multiplying all the time-dependent parameters by a correction factor:
\deqn{\exp\left(\frac{T_A}{T_1} - \frac{T_A}{T}\right)}{}
For a more detailed modeling one can multiply with an extra fraction \eqn{s(T_1)/s(T)}{} with (3 parameters):
\deqn{s(T) = 1 + \exp\left(\frac{T_{AL}}{T} - \frac{T_{AL}}{T_L}\right)}{}
or (5 parameters)
\deqn{s(T) = 1 + \exp\left(\frac{T_{AL}}{T} - \frac{T_{AL}}{T_L}\right) + \exp\left(\frac{T_{AH}}{T_H} - \frac{T_{AH}}{T}\right)}{}
\end{Details}
%
\begin{Value}
vector with temperature correction factors that affect all rates
\end{Value}
%
\begin{Examples}
\begin{ExampleCode}
tempcorr(c(330, 331, 332), 320, T_A = 12000, T_L = 277, T_H = 331, T_AL = 20000, T_AH = 190000)
\end{ExampleCode}
\end{Examples}
\printindex{}
\end{document}
